\documentclass{article}
\usepackage{ctex}
\usepackage{xcolor}
\usepackage{tikz}
\usepackage{amsmath,amssymb,amsfonts}%equation need this 
\usepackage{bm}
\usepackage{listings}
\usepackage{url}
\usepackage{hyperref}
\usepackage{color}
\usepackage{graphicx} %插入图片的宏包
\usepackage{float} %设置图片浮动位置的宏包
\usepackage{subfigure} %插入多图时用子图显示的宏包
\usepackage[a4paper, left = 10mm, right = 10mm, top = 15mm, bottom = 15mm]{geometry}

\newcommand{\red}[1]{{\color{red}{#1}}}
\newcommand{\green}[1]{{\color{green}{#1}}}
\newcommand{\yellow}[1]{{\color{yellow}{#1}}}
\newcommand{\orange}[1]{{\color{orange}{#1}}}
\newcommand{\blue}[1]{{\color{blue}{#1}}}
\newcommand{\pink}[1]{{\color{pink}{#1}}}
\newcommand{\tiffany}[1]{{\color{tiffany}{#1}}}
\newcommand{\purple}[1]{{\color{purple}{#1}}}
\newcommand{\darkgray}[1]{{\color{darkgray}{#1}}}
\newcommand{\p}{\par }
\newcommand{\pl}{\par \ \par}
\newcommand{\s}[1]{${#1}$}

\definecolor{tiffany}{RGB}{127, 208, 201}
\definecolor{lightgray}{rgb}{.9,.9,.9}
\definecolor{darkgray}{rgb}{.4,.4,.4}
\definecolor{purple}{rgb}{0.65, 0.12, 0.82}
 

\lstdefinelanguage{JavaScript}{
  keywords={typeof, new, true, false, catch, function, return, null, catch, switch, var, if, in, while, do, else, case, break},
  keywordstyle=\color{blue}\bfseries,
  ndkeywords={class, export, boolean, throw, implements, import, this},
  ndkeywordstyle=\color{darkgray}\bfseries,
  identifierstyle=\color{black},
  sensitive=false,
  comment=[l]{//},
  morecomment=[s]{/*}{*/},
  commentstyle=\color{purple}\ttfamily,
  stringstyle=\color{red}\ttfamily,
  morestring=[b]',
  morestring=[b]"
}

\lstset{
   language=JavaScript,
   extendedchars=true,
   basicstyle=\footnotesize\ttfamily,
   showstringspaces=false,
   showspaces=false,
%    numbers=left,
%    numberstyle=\footnotesize,
%    numbersep=9pt,
   tabsize=2,
   breaklines=true,
   showtabs=false,
   captionpos=b
}


\title{Vue笔记}
\author{曹思远}
\date{\today}
\begin{document}
\maketitle
\tableofcontents
\newpage
\begin{abstract}
    vue笔记
\end{abstract}

\section{起手式}
\subsection{Vue自学路线图}
\begin{figure}[H] %H为当前位置,!htb为忽略美学标准,htbp为浮动图形
    \centering %图片居中
    \includegraphics[width=0.7\textwidth]{image/vuelearn.png} %插入图片,[]中设置图片大小,{}中是图片文件名
    \caption{Vue自学路线图} %最终文档中希望显示的图片标题
\end{figure}
\subsection{项目搭建}
\red{搞出一个使用Vue的项目}
\subsubsection{两种方法}
\begin{enumerate}
    \item \orange{方法一: 使用@vue/cli}
          \begin{itemize}
              \item 搜索@vue/cli,进入官网
              \item 打开文档,打开创建一个项目章节
          \end{itemize}
    \item \orange{方法二: 自己从零搭建Vue项目}
          \begin{itemize}
              \item 使用webpack或者rollup从零开始
              \item 适合老手
          \end{itemize}
\end{enumerate}
\subsubsection{@vue/cli用法}
\begin{enumerate}
    \item 全局安装:yarn global add @vue/cli
    \item 创建目录:vue create路径 (路径可以用.点)
    \item 选择使用哪些配置
    \item 进入目录,运行yarn serve开启webpack-dev-server
    \item 用WebStorm或VSCode打开项目开始干
    \item 进入@vue/cli官网看目录
\end{enumerate}
\subsubsection{安装Vue的选项}
\begin{enumerate}
    \item 选错了请Ctrl+C中断然后重来
    \item 没有截图的都使用默认选项
    \item 真实项目自行斟酌
          \begin{figure}[H] %H为当前位置,!htb为忽略美学标准,htbp为浮动图形
              \centering %图片居中
              \includegraphics[width=0.7\textwidth]{image/vue0.png} %插入图片,[]中设置图片大小,{}中是图片文件名
              %\caption{Vue自学路线图} %最终文档中希望显示的图片标题
          \end{figure}
          \begin{figure}[H] %H为当前位置,!htb为忽略美学标准,htbp为浮动图形
              \centering %图片居中
              \includegraphics[width=0.7\textwidth]{image/vue1.png} %插入图片,[]中设置图片大小,{}中是图片文件名
              %\caption{Vue自学路线图} %最终文档中希望显示的图片标题
          \end{figure}
          \begin{figure}[H] %H为当前位置,!htb为忽略美学标准,htbp为浮动图形
              \centering %图片居中
              \includegraphics[width=0.7\textwidth]{image/vue3.png} %插入图片,[]中设置图片大小,{}中是图片文件名
              %\caption{Vue自学路线图} %最终文档中希望显示的图片标题
          \end{figure}
    \item \yellow{\underline{\href{https://github.com/syuancao/FrontNotePractice/tree/master/VUE/vue-demo-0}{vue demo地址}}}
\end{enumerate}
\subsection{Vue实例}
\red{很重要,做项目必须用到}
\subsubsection{如何使用Vue实例}
\begin{enumerate}
    \item \orange{方法一: 从HTML得到视图}
          \begin{itemize}
              \item 也就是文档里说的完整版Vue
              \item 从CDN引入vue.js或vue.min.js
              \item 也可以通过import引入vue.js或者vue.min.js
              \item 完整版视图支持从html引入
              \item 也可以用template写在js里面
              \item 完整版不好的地方是给用户的体积变大了
          \end{itemize}
    \item \orange{方法二: 用JS构建视图}
          \begin{itemize}
              \item 使用vue.runtime.js,也就是非完整版
              \item 不支持从html里获得视图,template也不行
              \item 这种方法不方便,但实际是对的
              \item 必须要用render(createElement)的方式把所有元素构造出来
                    \begin{lstlisting}
                new Vue({
                    el: "#app",
                    render(createElement) {
                        const h = createElement;
                        return h('div', [this.n, h('button', {
                            on:{click:this.add}, '+1'
                        })])
                    },
                    data: {
                        n: 0
                    },
                    methods: {
                        add() {
                            this.n += 1
                        }
                    }
                })
            \end{lstlisting}
              \item 好处是更加的独立,不需要编译器,减少百分之30的体积
          \end{itemize}
          \begin{figure}[H] %H为当前位置,!htb为忽略美学标准,htbp为浮动图形
              \centering %图片居中
              \includegraphics[width=0.7\textwidth]{image/runtimevue.png} %插入图片,[]中设置图片大小,{}中是图片文件名
              \caption{完整版与不完整版的区别} %最终文档中希望显示的图片标题
          \end{figure}
    \item \orange{方法三:使用vue-loader}
          \begin{itemize}
              \item 可以把.vue文件以及其template里的东西翻译成h构建方法
              \item 但这样做HTML就只有一个div\#app,SEO不友好
              \item 通过webpack让用户写的时候是完整版,但实际下载的时候是非完整版
          \end{itemize}
          \begin{figure}[H] %H为当前位置,!htb为忽略美学标准,htbp为浮动图形
              \centering %图片居中
              \includegraphics[width=0.7\textwidth]{image/webpackvue.png} %插入图片,[]中设置图片大小,{}中是图片文件名
              \caption{用webpack通过vue-loader转化减少代码体积} %最终文档中希望显示的图片标题
          \end{figure}
\end{enumerate}
\subsubsection{Vue实例的作用}
\begin{enumerate}
    \item \orange{Vue实例就如同jQuery实例}
          \begin{itemize}
              \item 封装了对DOM的所有操作
              \item 封装了对data的所有操作
          \end{itemize}
    \item \orange{操作DOM}
          \begin{itemize}
              \item 无非就是监听事件,改变DOM
          \end{itemize}
    \item \orange{操作data}
          \begin{itemize}
              \item 无非就是增删改查
              \item Vue2还有一个bug(面试考,在后面响应式原理里)
          \end{itemize}
    \item \orange{没有封装ajax}
          \begin{itemize}
              \item 用axios的ajax功能
          \end{itemize}
\end{enumerate}
\subsubsection{SEO基本原理}
\begin{enumerate}
    \item \orange{seo友好}
          \begin{itemize}
              \item 搜索引擎优化
              \item 你可以认为搜索引擎就是不停地curl
              \item 搜索引擎根据curl结果猜测页面内容
              \item 如果你的页面都是用JS创建的div,那么curl就很瞎
          \end{itemize}
    \item \orange{那怎么办}
          \begin{itemize}
              \item 给curl一点内容
              \item 把title,description, keyword, h1, a写好即可
              \item 原则:让curl能得到页面的信息,seo就能正常工作
          \end{itemize}
\end{enumerate}
\subsection{理解两种vue的区别}
\subsubsection{深入理解}
\begin{figure}[H] %H为当前位置,!htb为忽略美学标准,htbp为浮动图形
    \centering %图片居中
    \includegraphics[width=0.7\textwidth]{image/diffvue.png} %插入图片,[]中设置图片大小,{}中是图片文件名
    \caption{两种vue版本的区别} %最终文档中希望显示的图片标题
\end{figure}
\subsubsection{引用错了会怎样}
\begin{enumerate}
    \item \orange{vue.js错用成了vue.runtime.js}
          \begin{itemize}
              \item 无法将HTML编译成视图
          \end{itemize}
    \item \orange{vue.runtime.js错用成vue.js}
          \begin{itemize}
              \item 代码体积变大,因为vue.js有编译HTML的功能
          \end{itemize}
\end{enumerate}
\section{构造选项}
\subsection{new Vue里有什么}
\begin{figure}[H] %H为当前位置,!htb为忽略美学标准,htbp为浮动图形
    \centering %图片居中
    \includegraphics[width=0.7\textwidth]{image/vuememory.png} %插入图片,[]中设置图片大小,{}中是图片文件名
    \caption{实例的内存图} %最终文档中希望显示的图片标题
\end{figure}
\subsection{options里有什么}
\subsubsection{options里的五类属性}
\begin{enumerate}
    \item \orange{数据}
          \begin{itemize}
              \item \red{data, props}, \blue{propsData}, \yellow{computed}, \red{methods}, \yellow{watch}
          \end{itemize}
    \item \orange{DOM}
          \begin{itemize}
              \item \red{el}, \purple{template}, \blue{render}, \green{renderError}
          \end{itemize}
    \item \orange{生命周期钩子}
          \begin{itemize}
              \item \green{beforeCreate}, \red{created}, \green{beforeMount}, \red{mounted}, \green{beforeUpdate}, \red{updated}
              \item \yellow{activated}, \yellow{deactivated}, \green{beforeDestroy},  \red{destroyed},  errorCaptured
          \end{itemize}
    \item \orange{资源}
          \begin{itemize}
              \item \yellow{directives}, \purple{filters}, \red{components}
          \end{itemize}
    \item \orange{组合}
          \begin{itemize}
              \item parent, \yellow{mixins, extends, provide, inject}
          \end{itemize}
\end{enumerate}
\subsubsection{属性分段}
\begin{enumerate}
    \item \orange{红色属性(9)}
          \begin{itemize}
              \item 好学,必学,几句话就能说明白
          \end{itemize}
    \item \orange{黄色属性(9)}
          \begin{itemize}
              \item 高级属性,稍微费点力
          \end{itemize}
    \item \orange{绿色属性(5)}
          \begin{itemize}
              \item 简单
          \end{itemize}
    \item \orange{蓝色属性(2)}
          \begin{itemize}
              \item 不常用,可学可不学
          \end{itemize}
    \item \orange{紫色属性(2)}
          \begin{itemize}
              \item 比较特殊,重点
          \end{itemize}
    \item \orange{黑色属性(3)}
          \begin{itemize}
              \item 很不常用,用的时候看一下文档
          \end{itemize}
\end{enumerate}
\subsubsection{入门属性}
\begin{enumerate}
    \item \orange{el - 挂载点}
          \begin{itemize}
              \item 可以用\$mount代替
          \end{itemize}
    \item \orange{data - 内部数据}
          \begin{itemize}
              \item 支持对象和函数,优先用函数
          \end{itemize}
    \item \orange{methods - 方法}
          \begin{itemize}
              \item 事件处理函数或者是普通函数
          \end{itemize}
    \item \orange{components}
          \begin{itemize}
              \item Vue组件,注意大小写
              \item 三种引入方式,优先使用模块化
          \end{itemize}
    \item \orange{四个钩子}
          \begin{itemize}
              \item created - 实例出现在内存中
              \item mounted - 实例出现在页面中
              \item updated - 实例更新了
              \item destroyed - 实例从页面和内存中消亡了(可用切换visible演示)
          \end{itemize}
    \item \orange{props - 外部数据}
          \begin{itemize}
              \item 也叫属性
              \item message='n'传入字符串
              \item :message='n'传入this.n数据
              \item :fn='add'传入this.add函数
          \end{itemize}
\end{enumerate}
\section{数据响应式}
\section{computed和watch}
\section{模板,指令与修饰符}
\section{进阶构造属性}
\section{表单与v-model}
\section{Vue Router-前端路由实现思路}
\section{深入理解Vue动画原理}
\end{document}