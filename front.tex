\documentclass{article}
\usepackage{ctex}
\usepackage{graphicx}
\usepackage{xcolor}
\usepackage{tikz}
\usepackage{amsmath,amssymb,amsfonts}%equation need this 
\usepackage{bm}
\usepackage{listings}
\usepackage{url}
\newcommand{\red}[1]{{\color{red}{#1}}}
\newcommand{\green}[1]{{\color{green}{#1}}}
\newcommand{\p}{\par }
\newcommand{\pl}{\par \ \par}
\newcommand{\s}[1]{${#1}$}

\title{前端基础笔记}
\author{曹思远}
\date{\today}
\usepackage[a4paper, left = 10mm, right = 10mm, top = 15mm, bottom = 15mm]{geometry}

\begin{document}
\maketitle
\tableofcontents
\newpage
\begin{abstract}
    前端基础笔记
\end{abstract}

\section{软件安装}
\subsection{必备软件列表}
\begin{enumerate}
    \item vscode
    \item cmder
    \item chrome
    \item clash
    \item NodeJS, Yarn
\end{enumerate}
\subsubsection{必备软件配置}
\red{任何软件都需要配置}
\paragraph{vscode}
\begin{enumerate}
    \item 必备插件
          \begin{itemize}
              \item Chinese (Simplified) Language Pack for Visual Studio Code
              \item Code Spell Checker
              \item Git Easy
              \item Latex Workshop
              \item Markdown All in One
              \item Prettier - Code formatter
          \end{itemize}
    \item 环境和快捷键配置
          \begin{itemize}
              \item 具体看我知乎保存的setting.json 和 keybinding.json
          \end{itemize}
    \item 快捷键
          \begin{itemize}
              \item ctrl + p - 找文件
              \item ctrl + shift + p 或 F1 - 输命令
              \item alt + 单击 - 多位置输入
          \end{itemize}
\end{enumerate}
\paragraph{cmder}
\begin{enumerate}
    \item 配置内容较多,直接看我保存的配置文件。
    \item 将cmder加入右键菜单,加入环境变量。
\end{enumerate}
\paragraph{chrome}
\begin{enumerate}
    \item 可选插件
          \begin{itemize}
              \item Proxy SwitchOmega
              \item uBlock
          \end{itemize}
    \item 高级用户配置
          \begin{itemize}
              \item 在开发者工具\red{里面}按ESC可以新建控制台
              \item Sources面板可以保存代码片段(Snippets)
              \item Network关掉show overview,filter可以搜素,右击勾选Method
              \item Network可模拟慢网速/断网
              \item Preserve log不会清空当前请求数据
              \item Disable cache清除缓存
          \end{itemize}
    \item chrome常用快捷键
          \begin{itemize}
              \item 鼠标中键单击 - 打开或关闭
              \item ctrl + T - 新开标签
              \item ctrl + shift + T - 撤销关闭
              \item ctrl + 点击 - 在新标签打开
              \item ctrl + W - 关闭当前标签
              \item ctrl + Reload 或者 F5 - 刷新
              \item ctrl + Location - 输入网址
              \item ctrl + shift + Inspector 或 F12 - 打开开发者工具
              \item alt + 左右 - 前进后退
              \item alt + 回车 - 在新标签打开
              \item shift + 回车 - 在新窗口打开
              \item ctrl + shift + delete - 删除历史浏览数据
          \end{itemize}
\end{enumerate}
\paragraph{windows}
\begin{enumerate}
    \item 关掉任务栏无用标签,卸载无用软件。(如搜索框,任务视图,微软小娜Cortana,开始菜单里的各种贴图)
    \item 可安装TranslucentTB使任务栏透明
    \item 理解用户目录,即 C:\verb|\|Users\verb|\|Jony,分别右击用户目录里的下载和文件,属性-->位置-->移动到E盘。
    \item 显示文件后缀,打开查看-->选项-->查看,去掉隐藏已知文件扩展名,勾选显示已知文件,文件夹和驱动器。
    \item 记住组合键\begin{itemize}
              \item Win组合键
                    \begin{enumerate}
                        \item win + Desktop - 展示桌面
                        \item win + 方向键 - 移动窗口
                        \item alt + tab - 切换窗口
                        \item win + tab - 不怎么常用的切换窗口
                        \item win + ctrl + 方向键 - 切换桌面
                    \end{enumerate}
              \item Ctrl组合键
                    \begin{enumerate}
                        \item ctrl + All/ctrl + Copy/ctrl + V/ctrl + Z/ctrl + Y
                        \item ctrl + Reload/F5
                        \item ctrl + P - 打印
                    \end{enumerate}
          \end{itemize}
\end{enumerate}
\paragraph{NodeJS, Yarn}
\begin{enumerate}
    \item 开发必装的东西
          \begin{itemize}
              \item nrm
              \item tldr
          \end{itemize}
\end{enumerate}
\section{Git入门}
\subsection{命令行入门}
\subsection{本地仓库}
\subsection{远程仓库}
\section{HTML}
\subsection{概览}
\subsection{标签}
\subsection{重难点}
\subsection{实践和手机调试}
\section{CSS}
\subsection{基础}
\subsubsection{语法}
\subsubsection{Border调试法}
\subsubsection{文档流}
\begin{enumerate}
    \item 文档流的基本概念
          \begin{itemize}
              \item 流动方向
                    \begin{itemize}
                        \item inline元素从左到右,到达最右边才会换行
                        \item block元素从上到下,每一个都另起一行
                        \item inline-block也是从左到右
                    \end{itemize}
              \item 宽度
                    \begin{itemize}
                        \item inline宽度为内部inline元素的和,不能用width指定
                        \item block默认自动计算宽度,可用width指定
                        \item inline-block结合前两者特点,可用width
                    \end{itemize}
              \item 高度
                    \begin{itemize}
                        \item inline高度由line-height间接确定,跟height无关
                        \item block高度由文档流元素决定,可以设height
                        \item inline-block跟block类似,可以设置height
                    \end{itemize}
          \end{itemize}
    \item overflow 溢出
          \begin{itemize}
              \item 当内容大于容器
                    \begin{itemize}
                        \item 等内容的宽度或高度大于容器,会溢出
                        \item 可用overflow来设置是否显示滚动条
                        \item auto是灵活设置
                        \item scroll是永远显示
                        \item hidden是直接隐藏溢出部分
                        \item visible是直接显示溢出部分
                        \item overflow可以分为overflow-x和overflow-y
                    \end{itemize}
          \end{itemize}
    \item 脱离文档流
          \begin{itemize}
              \item 有些元素可不在文档流
                    \begin{itemize}
                        \item 原因是block高度由内部文档流元素决定,可以设height
                    \end{itemize}
              \item 以下元素脱离文档流
                    \begin{itemize}
                        \item float
                        \item position:absolute/fixed
                    \end{itemize}
              \item 不用以上属性就不脱离文档流
          \end{itemize}
\end{enumerate}
\subsubsection{盒模型}
\begin{enumerate}
    \item 两种
          \begin{itemize}
              \item content-box内容盒 - 内容就是盒子的边界
              \item border-box边框盒 - 边框才是盒子的边界
          \end{itemize}
    \item 公式
          \begin{itemize}
              \item content-box width = 内容宽度
              \item border-box width = 内容宽度 + padding + border
          \end{itemize}
    \item 哪个好用?
          \begin{itemize}
              \item border-box好用
              \item 因为可以同时指定padding, width, border
          \end{itemize}
\end{enumerate}
\subsubsection{margin合并}
\begin{enumerate}
    \item 哪些情况会合并
          \begin{itemize}
              \item 父子margin合并
              \item 兄弟margin合并
          \end{itemize}
    \item 如何阻止合并
          \begin{itemize}
              \item 父子合并用padding/border挡住
              \item 父子合并用overflow:hidden挡住
              \item 父子合并用display:flex
              \item 兄弟合并是符合预期的
              \item 兄弟合并可以用inline-block消除
              \item css属性逐年增多,每年都有新的,死记就完事了
          \end{itemize}
\end{enumerate}
\subsubsection{基本单位}
\begin{enumerate}
    \item 长度单位
          \begin{itemize}
              \item px像素
              \item em相对于自身font-size的倍数
              \item 百分数
              \item 整数
              \item rem
              \item vw和vh
          \end{itemize}
    \item 颜色
          \begin{itemize}
              \item 十六进制\#FF6600或者\#F60
              \item RGBA颜色rgb(0,0,0)或者rgba(0,0,0,1)
              \item hsl颜色hsl(360,100$\%$, 100$\%$)
          \end{itemize}
\end{enumerate}
\subsubsection{练手项目}
\red{彩虹demo}
\pl

\subsection{布局}
\subsubsection{布局分类}
\begin{enumerate}
    \item 两种
          \begin{itemize}
              \item 固定宽度布局,一般宽度为960/1000/1024px
              \item 不固定宽度布局,主要靠文档流的原来布局
          \end{itemize}
    \item 回顾
          \begin{itemize}
              \item 文档流本来就是自适应的,不需要加额外的样式
          \end{itemize}
    \item 响应式布局
          \begin{itemize}
              \item PC上固定宽度,手机上不固定宽度
              \item 也就是一种混合布局
          \end{itemize}
\end{enumerate}
\subsubsection{两种布局思路}
\begin{enumerate}
    \item 从大到小
          \begin{itemize}
              \item 先定下大局
              \item 然后完善每个部分的小布局
          \end{itemize}
    \item 从小到大
          \begin{itemize}
              \item 先完成小布局
              \item 然后组合成大布局
          \end{itemize}
    \item 两种均可
          \begin{itemize}
              \item 新人推荐第二种,因为小的简单
              \item 老手一般用第一种,因为熟练有大局观
          \end{itemize}
\end{enumerate}
\subsubsection{float布局}
一图流(图片以后贴出)
\begin{enumerate}
    \item float布局
          \begin{itemize}
              \item 子元素加上float:left和width
              \item \red{在父元素上加.clearfix}
          \end{itemize}
    \item float布局经验
          \begin{itemize}
              \item 留一些空间或者最后一个不设width
              \item 不需要做响应式,因为手机上没有IE,而这个布局是专门为IE准备的
              \item 解决IE6/7存在的双倍 margin bug如下
              \item 一是将错就错,针对IE6/7把margin减半
              \item 二是神来一笔,再加一个display:inline-block
          \end{itemize}
\end{enumerate}
\paragraph{float布局实践}
\begin{enumerate}
    \item 不同布局
          \begin{itemize}
              \item 用float做两栏布局(如顶部条)
              \item 用float做三栏布局(如内容区)
              \item 用float做四栏布局(如导航)
              \item 用float做平均布局(如产品展示区)
              \item
          \end{itemize}
    \item 实践经验
          \begin{itemize}
              \item 加上头尾,即可满足所有PC页面需求
              \item 手机页面傻子采用float
              \item float要程序员自己计算宽度,不灵活
              \item float用来应付IE足矣
          \end{itemize}
\end{enumerate}
\subsubsection{Flex布局}
\begin{enumerate}
    \item 重点
          \begin{itemize}
              \item display:flex
              \item flex-direction:row/column
              \item flex-wrap:wrap
              \item just-content:center/space-between
              \item align-item:center
          \end{itemize}
    \item 颜色
          \begin{itemize}
              \item 十六进制\#FF6600或者\#F60
              \item RGBA颜色rgb(0,0,0)或者rgba(0,0,0,1)
              \item hsl颜色hsl(360,100$\%$, 100$\%$)
          \end{itemize}
    \item 实践
          \begin{itemize}
              \item 用flex做两栏布局
              \item 用flex做三栏布局
              \item 用flex做四栏布局
              \item 用flex做平均布局
              \item 用flex组合使用,做更复杂的布局
              \item
          \end{itemize}
    \item 经验
          \begin{itemize}
              \item 永远不要把width和height写死,除非特殊说明
              \item 用min-width/max-width/min-height/max-height
              \item flex可以基本满足所有需求
              \item flex和margin-xxx:auto配合有意外的效果
          \end{itemize}
    \item 什么是写死
          \begin{itemize}
              \item width:100px
          \end{itemize}
    \item 不写死
          \begin{itemize}
              \item width:50\%
              \item max-width:100px
              \item width:30vw
              \item min-width:80\%
              \item 特点:不使用px,或者加min max前缀
          \end{itemize}
\end{enumerate}
\subsubsection{Grid布局}
\red{二维布局用Grid,一维布局用Flex}
\paragraph{语法}
\paragraph{例子和语法}
\subsection{定位}
\red{布局与定位的区别是:布局是屏幕平面上的,定位是垂直于屏幕的}
\subsubsection{一个div的分层}
\subsubsection{positon的五个取值}
\subsubsection{层叠上下文}
\subsection{动画}
\subsubsection{动画的原理}
\subsubsection{浏览器渲染的原理}
\paragraph{浏览器渲染过程}
\begin{enumerate}
    \item 根据HTML构建HTML树(DOM)
    \item 根据CSS构建CSS树(CSSOM)
    \item 将两颗树合并成一颗渲染树(render tree)
    \item Layout布局(文档流,盒模型,计算大小和位置)
    \item Paint绘制(把边框颜色,文字颜色,阴影等画出来)
    \item Compose合成(根据层叠关系展示画面)
\end{enumerate}
\paragraph{三棵树}
图片以后放
\paragraph{如何更新样式}
\red{一般我们采用JS来更新样式}
\begin{enumerate}
    \item 比如div.style.background='red'
    \item 比如div.style.display='none'
    \item 比如div.classList.add('red')
    \item 比如div.remove()直接删掉节点
\end{enumerate}
\paragraph{三种更新方式}
\begin{enumerate}
    \item JS/CSS > 样式 > 布局 > 绘制 > 合成
    \item JS/CSS > 样式 > 绘制 > 合成
    \item JS/CSS > 样式 > 合成
\end{enumerate}
\paragraph{三种更新方式区别}
\begin{enumerate}
    \item 第一种,全走
          \begin{itemize}
              \item div.remove()会触发当前消失,其他元素relayout
              \item
          \end{itemize}
    \item 第二种,跳过layout
          \begin{itemize}
              \item 改变背景颜色,直接repaint+composite
              \item
          \end{itemize}
    \item 第三种,跳过layout和paint
          \begin{itemize}
              \item 改变transform,只需composite
              \item 注意必须全屏查看效果,在iframe里看有问题
              \item
          \end{itemize}
\end{enumerate}
\subsubsection{CSS动画优化}
\paragraph{JS优化}
\begin{enumerate}
    \item 使用requestAnimationFrame代替setTimeout或setInterval
\end{enumerate}
\paragraph{JS优化}
\begin{enumerate}
    \item 使用will-change或translate
\end{enumerate}
\paragraph{参考文章}
\subsubsection{transition}
\red{位移translate\quad  缩放scale\quad 旋转rotate\quad  倾斜skew}
\paragraph{经验}
\begin{enumerate}
    \item 一般都不需要配合transition过度
    \item inline元素不支持transform,需要先变成block
\end{enumerate}
\paragraph{translate}
\begin{enumerate}
    \item 常用写法
          \begin{itemize}
              \item translateX(<length-percentage>)
              \item translateY(<length-percentage>)
              \item translate(<length-percentage>, <length-percentage>?)
              \item translateZ(<length>)且父容器perspective
              \item translate3d(x,y,z)
              \item 演示
          \end{itemize}
    \item 经验
          \begin{itemize}
              \item 看懂MDN语法示例
              \item translate(-50\%, -50\%)可做绝对定位元素的居中
          \end{itemize}
\end{enumerate}
\paragraph{scale}
\begin{enumerate}
    \item 常用写法
          \begin{itemize}
              \item scaleX(<number>)
              \item scaleX(<number>)
              \item scaleX(<number>, <number>?)
              \item 演示\url
          \end{itemize}
    \item 经验
          \begin{itemize}
              \item 用的少
          \end{itemize}
\end{enumerate}
\paragraph{rotate}
\begin{enumerate}
    \item 常用写法
          \begin{itemize}
              \item rotate([<angle>|<zero>])
              \item rotateZ([<angle>|<zero>])
              \item rotateX([<angle>|<zero>])
              \item rotateY([<angle>|<zero>])
              \item rotate3d太复杂
              \item 演示
          \end{itemize}
    \item 经验
          \begin{itemize}
              \item 一般用于360度选择制作loading
              \item 用到的时候查rotate MDN文档
          \end{itemize}
\end{enumerate}
\paragraph{skew}
\begin{enumerate}
    \item 常用写法
          \begin{itemize}
              \item skewX([<angle>|<zero>])
              \item skewY([<angle>|<zero>])
              \item skew([<angle>|<zero>],[<angle>|<zero>]?)
              \item 演示
          \end{itemize}
    \item 经验
          \begin{itemize}
              \item 用的较少
              \item 用到的时候查skew MDN文档
          \end{itemize}
\end{enumerate}
\paragraph{transform多重效果}
\begin{enumerate}
    \item 组合使用
          \begin{itemize}
              \item transform:scale(0.5) translate(-100\%, -100\%);
              \item transform:none;取消所有
          \end{itemize}
\end{enumerate}
\paragraph{参考文章}
\subsubsection{transition过渡}
\red{作用是补充中间帧}
\subsubsection{红心实践}
\red{css需要想象力}
\p
\section{HTTP}
\red{Hyper Text Transfer Protocol}
\subsection{URL}
\red{Uniform Resource Locator}
\pl
\red{协议+域名或IP+端口号+路径+查询字符串+锚点}
\subsubsection{IP}
\red{Internet Protocal}
\begin{enumerate}
    \item 约定了两件事
          \begin{itemize}
              \item 如何定位一台设备
              \item 如何封装数据报文,以跟其他设备交流
          \end{itemize}
    \item 外网IP
          \begin{itemize}
              \item 从电信租用带宽,一年一千多。
              \item 买了路由器,然后用电脑和手机分别连接路由器广播出来的无线WIFI。
              \item 路由器连上电信服务器,路由器有一个外围IP,这是你互联网中的地址。
              \item 重启路由器可能会被重新分配外围IP,也就是路由器没有固定的外网IP。
              \item 连接路由器的手机和电脑是内网IP。
          \end{itemize}
    \item 内网IP
          \begin{itemize}
              \item 路由器会在家里创建一个内网,内网设备使用内网IP,一般是192.169.xxx.xxx。
              \item 一般路由器会给自己分配一个好记的内网IP,如192.168.1.1。
              \item 然后路由器会给每一个内网中的设备分配不同的内网IP。
              \item  如电脑是192.168.1.2,手机是192.168.1.3。
          \end{itemize}
    \item 路由器的功能
          \begin{itemize}
              \item 现在路由器会有两个IP,一个外网IP和一个内网IP。
              \item 内网的设备可以互相访问,但是不能直接访问外网。
              \item 内网设备想要访问外围,就必须经过路由器中转。
              \item 外网中的设备可以互相访问,但是无法访问你的内网。
              \item 外网设备想要把内容送到内网,也必须通过路由器。
              \item 也就是说内网和外网就像两个隔绝的空间,无法互通,唯一的联通点就是路由器。
              \item 所以路由器有时候也被叫做网关。
          \end{itemize}
    \item 几个特殊的IP
          \begin{itemize}
              \item 127.0.0.1表示自己。
              \item localhost通过hosts指定为自己。
              \item 0.0.0.0不表示任何设备。
          \end{itemize}
\end{enumerate}
\subsubsection{端口}
\red{一台机器可以提供很多服务,每个服务一个号码,这个号码就叫端口号port}
\begin{enumerate}
    \item 一个比喻
          \begin{itemize}
              \item 麦当劳提供两个窗口,一号快餐,二号咖啡。
              \item 你去快餐窗口点咖啡会被拒绝,让你去两一个窗口。
              \item 你去咖啡窗口点快餐结果一样。
          \end{itemize}
    \item 一台机器可以提供不同服务
          \begin{itemize}
              \item 要提供HTTP服务最好使用80端口。
              \item 要提供HTTPS服务最好使用443端口。
              \item 要提供FTP服务最好使用21端口。
              \item 一共有65535个端口(基本够用)。
          \end{itemize}
    \item 端口使用规则
          \begin{itemize}
              \item 0到1023(2的10次方减1)号端口是留给系统使用的。
              \item 你只有拥有了管理员权限后,才能使用这1024个端口。
              \item 其他端口可以给普通用户使用。
              \item 比如http-server默认使用8080端口。
              \item 一个端口如果被占用,你就只能换一个端口。
          \end{itemize}
\end{enumerate}
\subsubsection{域名}
\begin{enumerate}
    \item 域名就是IP的别称
          \begin{itemize}
              \item baidu.com对应的什么IP --> ping baidu.com
              \item qq.com对应的什么IP --> ping qq.com
              \item 一个域名可以对应不同IP。
              \item 这个叫做均衡负载,防止一台机器扛不住。
              \item 一个IP可以对应不同域名。
              \item 这个叫做共享主机,穷开发者会这么做。
          \end{itemize}
    \item 域名和IP是怎么对应起来的
          \begin{itemize}
              \item 通过DNS
          \end{itemize}
    \item 当你输入qq.com的过程
          \begin{itemize}
              \item 你的Chrome浏览器会向电信提供的DNS服务器询问qq.com对应什么IP。
              \item 电信会回答一个IP(具体过程很发杂,不研究)。
              \item 然后Chrome才会向对应IP的80/443端口发送请求。
              \item 请求内容是查看qq.com的首页。
          \end{itemize}
    \item 为什么是80或443端口
          \begin{itemize}
              \item 服务器默认用80提供http服务。
              \item 服务器默认用443提供https服务。
              \item 可以在开发者工具看到具体的端口。
          \end{itemize}
    \item 题外话
          \begin{itemize}
              \item www.caosiyuan.com和caosiyuan.com不是同一域名。
              \item comn是顶级域名。
              \item caosiyuan.com是二级域名(俗称一级域名)。
              \item www.caosiyuan.com是三级域名(俗称二级)。
              \item 他们是父子关系
              \item 比如github.io把子域名xx.github.io免费给你使用
              \item 但www.caosiyuan.com和caosiyuan.com可以不是同一家公司,也可以是。
              \item www非常多余
          \end{itemize}
    \item 如何请求不同的页面
          \begin{itemize}
              \item 路径可以做到
              \item \url{https://developer.mozilla.org/zh-CN/docs/Web/HTML}
              \item \url{https://developer.mozilla.org/zh-CN/docs/Web/CSS}
              \item 使用chrome开发者工具Network面板看区别。
              \item 有点类似爬虫找规律。
          \end{itemize}
    \item 同一个页面,不同内容
          \begin{itemize}
              \item 查询参数可以做到
              \item \url{http://www.baidu.com/s?wd=hi}
              \item \url{http://www.baidu.com/s?wd=hello}
          \end{itemize}
    \item 同一个页面,不同位置
          \begin{itemize}
              \item 锚点可以做到
              \item \url{https://developer.mozilla.org/zh-CN/docs/Web/CSS#参考书}
              \item \url{https://developer.mozilla.org/zh-CN/docs/Web/CSS#教程}
              \item 注意,锚点看起来有中文,但实际不支持中文。
              \item \# 参考书会变成\#\s{\%E5\%8F\dots}。
              \item 锚点是无法在Network面板看到。
              \item 锚点不会传给服务器。
          \end{itemize}
\end{enumerate}
\subsubsection{HTTP协议}
\red{基于TCP和IP两个协议,规定了请求的格式是什么,响应的格式是什么}
\begin{enumerate}
    \item 用curl可以发HTTP请求
          \begin{itemize}
              \item curl -v http://baidu.com
              \item curl -s -v -- https://www.baidu.com
          \end{itemize}
    \item 理解一下概念
          \begin{itemize}
              \item url会被curl工具重写,先请求DNS获得IP
              \item 先进行TCP连接,TCP连接成功后,开始发送HTTP请求
              \item 请求内容看一眼
              \item 响应内容看一眼
              \item 响应结束后,关闭TCP连接(看不出来)
              \item 真正结束
          \end{itemize}
\end{enumerate}
\subsection{请求响应和NodeJS Sever}



\section{JS}
\section{算法与数据结构}
\section{JS编程接口}
\section{项目}
\section{MVC}
\section{Webpack}
\section{虚拟DOM与DOM diff}
\section{Vue}
\section{React}
\section{NodeJS}
\section{Vue3造轮子}

\end{document}