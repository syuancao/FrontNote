\documentclass{article}
\usepackage{ctex}
\usepackage{graphicx}
\usepackage{xcolor}
\usepackage{tikz}
\usepackage{amsmath,amssymb,amsfonts}%equation need this 
\usepackage{bm}
\usepackage{listings}
\usepackage{url}
\newcommand{\red}[1]{{\color{red}{#1}}}
\newcommand{\green}[1]{{\color{green}{#1}}}
\newcommand{\p}{\par }
\newcommand{\pl}{\par \ \par}
\newcommand{\s}[1]{${#1}$}

\title{前端基础笔记}
\author{曹思远}
\date{\today}
\usepackage[a4paper, left = 10mm, right = 10mm, top = 15mm, bottom = 15mm]{geometry}

\begin{document}
\maketitle
\tableofcontents
\newpage
\begin{abstract}
    前端基础笔记
\end{abstract}

\section{软件安装}
\subsection{必备软件列表}
\begin{enumerate}
    \item vscode
    \item cmder
    \item chrome
    \item clash
    \item NodeJS, Yarn
\end{enumerate}
\subsubsection{必备软件配置}
\red{任何软件都需要配置}
\paragraph{vscode}
\begin{enumerate}
    \item 必备插件
          \begin{itemize}
              \item Chinese (Simplified) Language Pack for Visual Studio Code
              \item Code Spell Checker
              \item Git Easy
              \item Latex Workshop
              \item Markdown All in One
              \item Prettier - Code formatter
          \end{itemize}
    \item 环境和快捷键配置
          \begin{itemize}
              \item 具体看我知乎保存的setting.json 和 keybinding.json
          \end{itemize}
    \item 快捷键
          \begin{itemize}
              \item ctrl + p - 找文件
              \item ctrl + shift + p 或 F1 - 输命令
              \item alt + 单击 - 多位置输入
          \end{itemize}
\end{enumerate}
\paragraph{cmder}
\begin{enumerate}
    \item 配置内容较多,直接看我保存的配置文件。
    \item 将cmder加入右键菜单,加入环境变量。
\end{enumerate}
\paragraph{chrome}
\begin{enumerate}
    \item 可选插件
          \begin{itemize}
              \item Proxy SwitchOmega
              \item uBlock
          \end{itemize}
    \item 高级用户配置
          \begin{itemize}
              \item 在开发者工具\red{里面}按ESC可以新建控制台
              \item Sources面板可以保存代码片段(Snippets)
              \item Network关掉show overview,filter可以搜素,右击勾选Method
              \item Network可模拟慢网速/断网
              \item Preserve log不会清空当前请求数据
              \item Disable cache清除缓存
          \end{itemize}
    \item chrome常用快捷键
          \begin{itemize}
              \item 鼠标中键单击 - 打开或关闭
              \item ctrl + T - 新开标签
              \item ctrl + shift + T - 撤销关闭
              \item ctrl + 点击 - 在新标签打开
              \item ctrl + W - 关闭当前标签
              \item ctrl + Reload 或者 F5 - 刷新
              \item ctrl + Location - 输入网址
              \item ctrl + shift + Inspector 或 F12 - 打开开发者工具
              \item alt + 左右 - 前进后退
              \item alt + 回车 - 在新标签打开
              \item shift + 回车 - 在新窗口打开
              \item ctrl + shift + delete - 删除历史浏览数据
          \end{itemize}
\end{enumerate}
\paragraph{windows}
\begin{enumerate}
    \item 关掉任务栏无用标签,卸载无用软件。(如搜索框,任务视图,微软小娜Cortana,开始菜单里的各种贴图)
    \item 可安装TranslucentTB使任务栏透明
    \item 理解用户目录,即 C:\verb|\|Users\verb|\|Jony,分别右击用户目录里的下载和文件,属性-->位置-->移动到E盘。
    \item 显示文件后缀,打开查看-->选项-->查看,去掉隐藏已知文件扩展名,勾选显示已知文件,文件夹和驱动器。
    \item 记住组合键\begin{itemize}
              \item Win组合键
                    \begin{enumerate}
                        \item win + Desktop - 展示桌面
                        \item win + 方向键 - 移动窗口
                        \item alt + tab - 切换窗口
                        \item win + tab - 不怎么常用的切换窗口
                        \item win + ctrl + 方向键 - 切换桌面
                    \end{enumerate}
              \item Ctrl组合键
                    \begin{enumerate}
                        \item ctrl + All/ctrl + Copy/ctrl + V/ctrl + Z/ctrl + Y
                        \item ctrl + Reload/F5
                        \item ctrl + P - 打印
                    \end{enumerate}
          \end{itemize}
\end{enumerate}
\paragraph{NodeJS, Yarn}
\begin{enumerate}
    \item 开发必装的东西
          \begin{itemize}
              \item nrm
              \item tldr
          \end{itemize}
\end{enumerate}
\section{Git入门}
\subsection{命令行入门}
\subsection{本地仓库}
\subsection{远程仓库}
\section{HTML}
\subsection{概览}
\subsection{标签}
\subsection{重难点}
\subsection{实践和手机调试}
\section{CSS}
\subsection{基础}
\subsubsection{语法}
\subsubsection{Border调试法}
\subsubsection{文档流}
\begin{enumerate}
    \item 文档流的基本概念
          \begin{itemize}
              \item 流动方向
                    \begin{itemize}
                        \item inline元素从左到右,到达最右边才会换行
                        \item block元素从上到下,每一个都另起一行
                        \item inline-block也是从左到右
                    \end{itemize}
              \item 宽度
                    \begin{itemize}
                        \item inline宽度为内部inline元素的和,不能用width指定
                        \item block默认自动计算宽度,可用width指定
                        \item inline-block结合前两者特点,可用width
                    \end{itemize}
              \item 高度
                    \begin{itemize}
                        \item inline高度由line-height间接确定,跟height无关
                        \item block高度由文档流元素决定,可以设height
                        \item inline-block跟block类似,可以设置height
                    \end{itemize}
          \end{itemize}
    \item overflow 溢出
          \begin{itemize}
              \item 当内容大于容器
                    \begin{itemize}
                        \item 等内容的宽度或高度大于容器,会溢出
                        \item 可用overflow来设置是否显示滚动条
                        \item auto是灵活设置
                        \item scroll是永远显示
                        \item hidden是直接隐藏溢出部分
                        \item visible是直接显示溢出部分
                        \item overflow可以分为overflow-x和overflow-y
                    \end{itemize}
          \end{itemize}
    \item 脱离文档流
          \begin{itemize}
              \item 有些元素可不在文档流
                    \begin{itemize}
                        \item 原因是block高度由内部文档流元素决定,可以设height
                    \end{itemize}
              \item 以下元素脱离文档流
                    \begin{itemize}
                        \item float
                        \item position:absolute/fixed
                    \end{itemize}
              \item 不用以上属性就不脱离文档流
          \end{itemize}
\end{enumerate}
\subsubsection{盒模型}
\begin{enumerate}
    \item 两种
          \begin{itemize}
              \item content-box内容盒 - 内容就是盒子的边界
              \item border-box边框盒 - 边框才是盒子的边界
          \end{itemize}
    \item 公式
          \begin{itemize}
              \item content-box width = 内容宽度
              \item border-box width = 内容宽度 + padding + border
          \end{itemize}
    \item 哪个好用?
          \begin{itemize}
              \item border-box好用
              \item 因为可以同时指定padding, width, border
          \end{itemize}
\end{enumerate}
\subsubsection{margin合并}
\begin{enumerate}
    \item 哪些情况会合并
          \begin{itemize}
              \item 父子margin合并
              \item 兄弟margin合并
          \end{itemize}
    \item 如何阻止合并
          \begin{itemize}
              \item 父子合并用padding/border挡住
              \item 父子合并用overflow:hidden挡住
              \item 父子合并用display:flex
              \item 兄弟合并是符合预期的
              \item 兄弟合并可以用inline-block消除
              \item css属性逐年增多,每年都有新的,死记就完事了
          \end{itemize}
\end{enumerate}
\subsubsection{基本单位}
\begin{enumerate}
    \item 长度单位
          \begin{itemize}
              \item px像素
              \item em相对于自身font-size的倍数
              \item 百分数
              \item 整数
              \item rem
              \item vw和vh
          \end{itemize}
    \item 颜色
          \begin{itemize}
              \item 十六进制\#FF6600或者\#F60
              \item RGBA颜色rgb(0,0,0)或者rgba(0,0,0,1)
              \item hsl颜色hsl(360,100$\%$, 100$\%$)
          \end{itemize}
\end{enumerate}
\subsubsection{练手项目}
\red{彩虹demo}
\pl
\url{http://js.jirengu.com/xacet/1/edit?html,css,output}
\subsection{布局}
\subsubsection{布局分类}
\begin{enumerate}
    \item 两种
          \begin{itemize}
              \item 固定宽度布局,一般宽度为960/1000/1024px
              \item 不固定宽度布局,主要靠文档流的原来布局
          \end{itemize}
    \item 回顾
          \begin{itemize}
              \item 文档流本来就是自适应的,不需要加额外的样式
          \end{itemize}
    \item 响应式布局
          \begin{itemize}
              \item PC上固定宽度,手机上不固定宽度
              \item 也就是一种混合布局
          \end{itemize}
\end{enumerate}
\subsubsection{两种布局思路}
\begin{enumerate}
    \item 从大到小
          \begin{itemize}
              \item 先定下大局
              \item 然后完善每个部分的小布局
          \end{itemize}
    \item 从小到大
          \begin{itemize}
              \item 先完成小布局
              \item 然后组合成大布局
          \end{itemize}
    \item 两种均可
          \begin{itemize}
              \item 新人推荐第二种,因为小的简单
              \item 老手一般用第一种,因为熟练有大局观
          \end{itemize}
\end{enumerate}
\subsubsection{布局需要的属性即布局套路}

\subsection{定位}
\subsection{动画}
\section{HTTP}
\section{JS}
\section{算法与数据结构}
\section{JS编程接口}
\section{项目}
\section{MVC}
\section{Webpack}
\section{虚拟DOM与DOM diff}
\section{Vue}
\section{React}
\section{NodeJS}
\section{Vue3造轮子}

\end{document}