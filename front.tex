\documentclass{article}
\usepackage{ctex}
\usepackage{graphicx}
\usepackage{xcolor}
\usepackage{tikz}
\usepackage{amsmath,amssymb,amsfonts}%equation need this 
\usepackage{bm}
\usepackage{listings}
\newcommand{\red}[1]{{\color{red}{#1}}}
\newcommand{\green}[1]{{\color{green}{#1}}}
\newcommand{\p}{\par }
\newcommand{\pl}{\par \ \par}
\newcommand{\s}[1]{${#1}$}

\title{前端基础笔记}
\author{曹思远}
\date{\today}
\usepackage[a4paper, left = 10mm, right = 10mm, top = 15mm, bottom = 15mm]{geometry}

\begin{document}
\maketitle
\tableofcontents
\newpage
\begin{abstract}
    前端基础笔记
\end{abstract}

\section{软件安装}
\subsection{必备软件列表}
\begin{enumerate}
    \item vscode
    \item cmder
    \item chrome
    \item clash
    \item NodeJS, Yarn
\end{enumerate}
\subsubsection{必备软件配置}
\red{任何软件都需要配置}
\paragraph{vscode}
\begin{enumerate}
    \item 必备插件
          \begin{itemize}
              \item Chinese (Simplified) Language Pack for Visual Studio Code
              \item Code Spell Checker
              \item Git Easy
              \item Latex Workshop
              \item Markdown All in One
              \item Prettier - Code formatter
          \end{itemize}
    \item 环境和快捷键配置
          \begin{itemize}
              \item 具体看我知乎保存的setting.json 和 keybinding.json
          \end{itemize}
    \item 快捷键
          \begin{itemize}
              \item ctrl + p - 找文件
              \item ctrl + shift + p 或 F1 - 输命令
              \item alt + 单击 - 多位置输入
          \end{itemize}
\end{enumerate}
\paragraph{cmder}
\begin{enumerate}
    \item 配置内容较多,直接看我保存的配置文件。
    \item 将cmder加入右键菜单,加入环境变量。
\end{enumerate}
\paragraph{chrome}
\begin{enumerate}
    \item 可选插件
          \begin{itemize}
              \item Proxy SwitchOmega
              \item uBlock
          \end{itemize}
    \item 高级用户配置
          \begin{itemize}
              \item 在开发者工具\red{里面}按ESC可以新建控制台
              \item Sources面板可以保存代码片段(Snippets)
              \item Network关掉show overview,filter可以搜素,右击勾选Method
              \item Network可模拟慢网速/断网
              \item Preserve log不会清空当前请求数据
              \item Disable cache清除缓存
          \end{itemize}
    \item chrome常用快捷键
          \begin{itemize}
              \item 鼠标中键单击 - 打开或关闭
              \item ctrl + T - 新开标签
              \item ctrl + shift + T - 撤销关闭
              \item ctrl + 点击 - 在新标签打开
              \item ctrl + W - 关闭当前标签
              \item ctrl + Reload 或者 F5 - 刷新
              \item ctrl + Location - 输入网址
              \item ctrl + shift + Inspector 或 F12 - 打开开发者工具
              \item alt + 左右 - 前进后退
              \item alt + 回车 - 在新标签打开
              \item shift + 回车 - 在新窗口打开
              \item ctrl + shift + delete - 删除历史浏览数据
          \end{itemize}
\end{enumerate}
\paragraph{windows}
\begin{enumerate}
    \item 关掉任务栏无用标签,卸载无用软件。(如搜索框,任务视图,微软小娜Cortana,开始菜单里的各种贴图)
    \item 可安装TranslucentTB使任务栏透明
    \item 理解用户目录,即 C:\verb|\|Users\verb|\|Jony,分别右击用户目录里的下载和文件,属性-->位置-->移动到E盘。
    \item 显示文件后缀,打开查看-->选项-->查看,去掉隐藏已知文件扩展名,勾选显示已知文件,文件夹和驱动器。
    \item 记住组合键\begin{itemize}
              \item Win组合键
                    \begin{enumerate}
                        \item win + Desktop - 展示桌面
                        \item win + 方向键 - 移动窗口
                        \item alt + tab - 切换窗口
                        \item win + tab - 不怎么常用的切换窗口
                        \item win + ctrl + 方向键 - 切换桌面
                    \end{enumerate}
              \item Ctrl组合键
                    \begin{enumerate}
                        \item ctrl + All/ctrl + Copy/ctrl + V/ctrl + Z/ctrl + Y
                        \item ctrl + Reload/F5
                        \item ctrl + P - 打印
                    \end{enumerate}
          \end{itemize}
\end{enumerate}
\paragraph{NodeJS, Yarn}
\begin{enumerate}
    \item 开发必装的东西
          \begin{itemize}
              \item nrm
              \item tldr
          \end{itemize}
\end{enumerate}
\section{Git入门}
\subsection{命令行入门}
\subsection{本地仓库}
\subsection{远程仓库}
\section{HTML全解}
\section{CSS}
\section{HTTP}
\section{JS}
\section{算法与数据结构}
\section{JS编程接口}
\section{项目}
\section{MVC}
\section{Webpack}
\section{虚拟DOM与DOM diff}
\section{Vue}
\section{React}
\section{NodeJS}
\section{Vue3造轮子}

\end{document}