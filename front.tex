\documentclass{article}
\usepackage{ctex}
\usepackage{xcolor}
\usepackage{tikz}
\usepackage{amsmath,amssymb,amsfonts}%equation need this 
\usepackage{bm}
\usepackage{listings}
\usepackage{url}
\usepackage{hyperref}
\usepackage{graphicx} %插入图片的宏包
\usepackage{float} %设置图片浮动位置的宏包
\usepackage{subfigure} %插入多图时用子图显示的宏包
\usepackage[a4paper, left = 10mm, right = 10mm, top = 15mm, bottom = 15mm]{geometry}
 
\newcommand{\red}[1]{{\color{red}{#1}}}
\newcommand{\green}[1]{{\color{green}{#1}}}
\newcommand{\p}{\par }
\newcommand{\pl}{\par \ \par}
\newcommand{\s}[1]{${#1}$}

\title{前端基础笔记}
\author{曹思远}
\date{\today}
\begin{document}
\maketitle
\tableofcontents
\newpage
\begin{abstract}
    前端基础笔记
\end{abstract}

\section{软件安装}
\subsection{必备软件列表}
\begin{enumerate}
    \item vscode
    \item cmder
    \item chrome
    \item clash
    \item NodeJS, Yarn
\end{enumerate}
\subsubsection{必备软件配置}
\red{任何软件都需要配置}
\paragraph{vscode}
\begin{enumerate}
    \item 必备插件
          \begin{itemize}
              \item Chinese (Simplified) Language Pack for Visual Studio Code
              \item Code Spell Checker
              \item Git Easy
              \item Latex Workshop
              \item Markdown All in One
              \item Prettier - Code formatter
          \end{itemize}
    \item 环境和快捷键配置
          \begin{itemize}
              \item 具体看我知乎保存的setting.json 和 keybinding.json
          \end{itemize}
    \item 快捷键
          \begin{itemize}
              \item ctrl + p - 找文件
              \item ctrl + shift + p 或 F1 - 输命令
              \item alt + 单击 - 多位置输入
          \end{itemize}
\end{enumerate}
\paragraph{cmder}
\begin{enumerate}
    \item 配置内容较多,直接看我保存的配置文件。
    \item 将cmder加入右键菜单,加入环境变量。
\end{enumerate}
\paragraph{chrome}
\begin{enumerate}
    \item 可选插件
          \begin{itemize}
              \item Proxy SwitchOmega
              \item uBlock
          \end{itemize}
    \item 高级用户配置
          \begin{itemize}
              \item 在开发者工具\red{里面}按ESC可以新建控制台
              \item Sources面板可以保存代码片段(Snippets)
              \item Network关掉show overview,filter可以搜素,右击勾选Method
              \item Network可模拟慢网速/断网
              \item Preserve log不会清空当前请求数据
              \item Disable cache清除缓存
          \end{itemize}
    \item chrome常用快捷键
          \begin{itemize}
              \item 鼠标中键单击 - 打开或关闭
              \item ctrl + T - 新开标签
              \item ctrl + shift + T - 撤销关闭
              \item ctrl + 点击 - 在新标签打开
              \item ctrl + W - 关闭当前标签
              \item ctrl + Reload 或者 F5 - 刷新
              \item ctrl + Location - 输入网址
              \item ctrl + shift + Inspector 或 F12 - 打开开发者工具
              \item alt + 左右 - 前进后退
              \item alt + 回车 - 在新标签打开
              \item shift + 回车 - 在新窗口打开
              \item ctrl + shift + delete - 删除历史浏览数据
          \end{itemize}
\end{enumerate}
\paragraph{windows}
\begin{enumerate}
    \item 关掉任务栏无用标签,卸载无用软件。(如搜索框,任务视图,微软小娜Cortana,开始菜单里的各种贴图)
    \item 可安装TranslucentTB使任务栏透明
    \item 理解用户目录,即 C:\verb|\|Users\verb|\|Jony,分别右击用户目录里的下载和文件,属性-->位置-->移动到E盘。
    \item 显示文件后缀,打开查看-->选项-->查看,去掉隐藏已知文件扩展名,勾选显示已知文件,文件夹和驱动器。
    \item 记住组合键\begin{itemize}
              \item Win组合键
                    \begin{enumerate}
                        \item win + Desktop - 展示桌面
                        \item win + 方向键 - 移动窗口
                        \item alt + tab - 切换窗口
                        \item win + tab - 不怎么常用的切换窗口
                        \item win + ctrl + 方向键 - 切换桌面
                    \end{enumerate}
              \item Ctrl组合键
                    \begin{enumerate}
                        \item ctrl + All/ctrl + Copy/ctrl + V/ctrl + Z/ctrl + Y
                        \item ctrl + Reload/F5
                        \item ctrl + P - 打印
                    \end{enumerate}
          \end{itemize}
\end{enumerate}
\paragraph{NodeJS, Yarn}
\begin{enumerate}
    \item 开发必装的东西
          \begin{itemize}
              \item nrm
              \item tldr
          \end{itemize}
\end{enumerate}
\section{Git入门}
\subsection{命令行入门}
\subsection{本地仓库}
\subsection{远程仓库}
\section{HTML}
\subsection{概览}
\subsection{标签}
\subsection{重难点}
\subsection{实践和手机调试}
\section{CSS}
\subsection{基础}
\subsubsection{语法}
\subsubsection{Border调试法}
\subsubsection{文档流}
\begin{enumerate}
    \item 文档流的基本概念
          \begin{itemize}
              \item 流动方向
                    \begin{itemize}
                        \item inline元素从左到右,到达最右边才会换行
                        \item block元素从上到下,每一个都另起一行
                        \item inline-block也是从左到右
                    \end{itemize}
              \item 宽度
                    \begin{itemize}
                        \item inline宽度为内部inline元素的和,不能用width指定
                        \item block默认自动计算宽度,可用width指定
                        \item inline-block结合前两者特点,可用width
                    \end{itemize}
              \item 高度
                    \begin{itemize}
                        \item inline高度由line-height间接确定,跟height无关
                        \item block高度由文档流元素决定,可以设height
                        \item inline-block跟block类似,可以设置height
                    \end{itemize}
          \end{itemize}
    \item overflow 溢出
          \begin{itemize}
              \item 当内容大于容器
                    \begin{itemize}
                        \item 等内容的宽度或高度大于容器,会溢出
                        \item 可用overflow来设置是否显示滚动条
                        \item auto是灵活设置
                        \item scroll是永远显示
                        \item hidden是直接隐藏溢出部分
                        \item visible是直接显示溢出部分
                        \item overflow可以分为overflow-x和overflow-y
                    \end{itemize}
          \end{itemize}
    \item 脱离文档流
          \begin{itemize}
              \item 有些元素可不在文档流
                    \begin{itemize}
                        \item 原因是block高度由内部文档流元素决定,可以设height
                    \end{itemize}
              \item 以下元素脱离文档流
                    \begin{itemize}
                        \item float
                        \item position:absolute/fixed
                    \end{itemize}
              \item 不用以上属性就不脱离文档流
          \end{itemize}
\end{enumerate}
\subsubsection{盒模型}
\begin{enumerate}
    \item 两种
          \begin{itemize}
              \item content-box内容盒 - 内容就是盒子的边界
              \item border-box边框盒 - 边框才是盒子的边界
          \end{itemize}
    \item 公式
          \begin{itemize}
              \item content-box width = 内容宽度
              \item border-box width = 内容宽度 + padding + border
          \end{itemize}
    \item 哪个好用?
          \begin{itemize}
              \item border-box好用
              \item 因为可以同时指定padding, width, border
          \end{itemize}
\end{enumerate}
\subsubsection{margin合并}
\begin{enumerate}
    \item 哪些情况会合并
          \begin{itemize}
              \item 父子margin合并
              \item 兄弟margin合并
          \end{itemize}
    \item 如何阻止合并
          \begin{itemize}
              \item 父子合并用padding/border挡住
              \item 父子合并用overflow:hidden挡住
              \item 父子合并用display:flex
              \item 兄弟合并是符合预期的
              \item 兄弟合并可以用inline-block消除
              \item css属性逐年增多,每年都有新的,死记就完事了
          \end{itemize}
\end{enumerate}
\subsubsection{基本单位}
\begin{enumerate}
    \item 长度单位
          \begin{itemize}
              \item px像素
              \item em相对于自身font-size的倍数
              \item 百分数
              \item 整数
              \item rem
              \item vw和vh
          \end{itemize}
    \item 颜色
          \begin{itemize}
              \item 十六进制\#FF6600或者\#F60
              \item RGBA颜色rgb(0,0,0)或者rgba(0,0,0,1)
              \item hsl颜色hsl(360,100$\%$, 100$\%$)
          \end{itemize}
\end{enumerate}
\subsubsection{练手项目}
\red{彩虹demo}
\pl
\subsection{布局}
\subsubsection{布局分类}
\begin{enumerate}
    \item 两种
          \begin{itemize}
              \item 固定宽度布局,一般宽度为960/1000/1024px
              \item 不固定宽度布局,主要靠文档流的原来布局
          \end{itemize}
    \item 回顾
          \begin{itemize}
              \item 文档流本来就是自适应的,不需要加额外的样式
          \end{itemize}
    \item 响应式布局
          \begin{itemize}
              \item PC上固定宽度,手机上不固定宽度
              \item 也就是一种混合布局
          \end{itemize}
\end{enumerate}
\subsubsection{两种布局思路}
\begin{enumerate}
    \item 从大到小
          \begin{itemize}
              \item 先定下大局
              \item 然后完善每个部分的小布局
          \end{itemize}
    \item 从小到大
          \begin{itemize}
              \item 先完成小布局
              \item 然后组合成大布局
          \end{itemize}
    \item 两种均可
          \begin{itemize}
              \item 新人推荐第二种,因为小的简单
              \item 老手一般用第一种,因为熟练有大局观
          \end{itemize}
\end{enumerate}
\subsubsection{float布局}
一图流(图片以后贴出)
\begin{enumerate}
    \item float布局
          \begin{itemize}
              \item 子元素加上float:left和width
              \item \red{在父元素上加.clearfix}
          \end{itemize}
    \item float布局经验
          \begin{itemize}
              \item 留一些空间或者最后一个不设width
              \item 不需要做响应式,因为手机上没有IE,而这个布局是专门为IE准备的
              \item 解决IE6/7存在的双倍 margin bug如下
              \item 一是将错就错,针对IE6/7把margin减半
              \item 二是神来一笔,再加一个display:inline-block
          \end{itemize}
\end{enumerate}
\paragraph{float布局实践}
\begin{enumerate}
    \item 不同布局
          \begin{itemize}
              \item 用float做两栏布局(如顶部条)
              \item 用float做三栏布局(如内容区)
              \item 用float做四栏布局(如导航)
              \item 用float做平均布局(如产品展示区)
              \item
          \end{itemize}
    \item 实践经验
          \begin{itemize}
              \item 加上头尾,即可满足所有PC页面需求
              \item 手机页面傻子采用float
              \item float要程序员自己计算宽度,不灵活
              \item float用来应付IE足矣
          \end{itemize}
\end{enumerate}
\subsubsection{Flex布局}
\begin{enumerate}
    \item 重点
          \begin{itemize}
              \item display:flex
              \item flex-direction:row/column
              \item flex-wrap:wrap
              \item just-content:center/space-between
              \item align-item:center
          \end{itemize}
    \item 颜色
          \begin{itemize}
              \item 十六进制\#FF6600或者\#F60
              \item RGBA颜色rgb(0,0,0)或者rgba(0,0,0,1)
              \item hsl颜色hsl(360,100$\%$, 100$\%$)
          \end{itemize}
    \item 实践
          \begin{itemize}
              \item 用flex做两栏布局
              \item 用flex做三栏布局
              \item 用flex做四栏布局
              \item 用flex做平均布局
              \item 用flex组合使用,做更复杂的布局
              \item
          \end{itemize}
    \item 经验
          \begin{itemize}
              \item 永远不要把width和height写死,除非特殊说明
              \item 用min-width/max-width/min-height/max-height
              \item flex可以基本满足所有需求
              \item flex和margin-xxx:auto配合有意外的效果
          \end{itemize}
    \item 什么是写死
          \begin{itemize}
              \item width:100px
          \end{itemize}
    \item 不写死
          \begin{itemize}
              \item width:50\%
              \item max-width:100px
              \item width:30vw
              \item min-width:80\%
              \item 特点:不使用px,或者加min max前缀
          \end{itemize}
\end{enumerate}
\subsubsection{Grid布局}
\red{二维布局用Grid,一维布局用Flex}
\paragraph{语法}
\paragraph{例子和语法}
\subsection{定位}
\red{布局与定位的区别是:布局是屏幕平面上的,定位是垂直于屏幕的}
\subsubsection{一个div的分层}
\subsubsection{positon的五个取值}
\subsubsection{层叠上下文}
\subsection{动画}
\subsubsection{动画的原理}
\subsubsection{浏览器渲染的原理}
\paragraph{浏览器渲染过程}
\begin{enumerate}
    \item 根据HTML构建HTML树(DOM)
    \item 根据CSS构建CSS树(CSSOM)
    \item 将两颗树合并成一颗渲染树(render tree)
    \item Layout布局(文档流,盒模型,计算大小和位置)
    \item Paint绘制(把边框颜色,文字颜色,阴影等画出来)
    \item Compose合成(根据层叠关系展示画面)
\end{enumerate}
\paragraph{三棵树}
图片以后放
\paragraph{如何更新样式}
\red{一般我们采用JS来更新样式}
\begin{enumerate}
    \item 比如div.style.background='red'
    \item 比如div.style.display='none'
    \item 比如div.classList.add('red')
    \item 比如div.remove()直接删掉节点
\end{enumerate}
\paragraph{三种更新方式}
\begin{enumerate}
    \item JS/CSS > 样式 > 布局 > 绘制 > 合成
    \item JS/CSS > 样式 > 绘制 > 合成
    \item JS/CSS > 样式 > 合成
\end{enumerate}
\paragraph{三种更新方式区别}
\begin{enumerate}
    \item 第一种,全走
          \begin{itemize}
              \item div.remove()会触发当前消失,其他元素relayout
              \item
          \end{itemize}
    \item 第二种,跳过layout
          \begin{itemize}
              \item 改变背景颜色,直接repaint+composite
              \item
          \end{itemize}
    \item 第三种,跳过layout和paint
          \begin{itemize}
              \item 改变transform,只需composite
              \item 注意必须全屏查看效果,在iframe里看有问题
              \item
          \end{itemize}
\end{enumerate}
\subsubsection{CSS动画优化}
\paragraph{JS优化}
\begin{enumerate}
    \item 使用requestAnimationFrame代替setTimeout或setInterval
\end{enumerate}
\paragraph{JS优化}
\begin{enumerate}
    \item 使用will-change或translate
\end{enumerate}
\paragraph{参考文章}
\subsubsection{transition}
\red{位移translate\quad  缩放scale\quad 旋转rotate\quad  倾斜skew}
\paragraph{经验}
\begin{enumerate}
    \item 一般都不需要配合transition过度
    \item inline元素不支持transform,需要先变成block
\end{enumerate}
\paragraph{translate}
\begin{enumerate}
    \item 常用写法
          \begin{itemize}
              \item translateX(<length-percentage>)
              \item translateY(<length-percentage>)
              \item translate(<length-percentage>, <length-percentage>?)
              \item translateZ(<length>)且父容器perspective
              \item translate3d(x,y,z)
              \item 演示
          \end{itemize}
    \item 经验
          \begin{itemize}
              \item 看懂MDN语法示例
              \item translate(-50\%, -50\%)可做绝对定位元素的居中
          \end{itemize}
\end{enumerate}
\paragraph{scale}
\begin{enumerate}
    \item 常用写法
          \begin{itemize}
              \item scaleX(<number>)
              \item scaleX(<number>)
              \item scaleX(<number>, <number>?)
              \item 演示
          \end{itemize}
    \item 经验
          \begin{itemize}
              \item 用的少
          \end{itemize}
\end{enumerate}
\paragraph{rotate}
\begin{enumerate}
    \item 常用写法
          \begin{itemize}
              \item rotate([<angle>|<zero>])
              \item rotateZ([<angle>|<zero>])
              \item rotateX([<angle>|<zero>])
              \item rotateY([<angle>|<zero>])
              \item rotate3d太复杂
              \item 演示
          \end{itemize}
    \item 经验
          \begin{itemize}
              \item 一般用于360度选择制作loading
              \item 用到的时候查rotate MDN文档
          \end{itemize}
\end{enumerate}
\paragraph{skew}
\begin{enumerate}
    \item 常用写法
          \begin{itemize}
              \item skewX([<angle>|<zero>])
              \item skewY([<angle>|<zero>])
              \item skew([<angle>|<zero>],[<angle>|<zero>]?)
              \item 演示
          \end{itemize}
    \item 经验
          \begin{itemize}
              \item 用的较少
              \item 用到的时候查skew MDN文档
          \end{itemize}
\end{enumerate}
\paragraph{transform多重效果}
\begin{enumerate}
    \item 组合使用
          \begin{itemize}
              \item transform:scale(0.5) translate(-100\%, -100\%);
              \item transform:none;取消所有
          \end{itemize}
\end{enumerate}
\paragraph{参考文章}
\subsubsection{transition过渡}
\red{作用是补充中间帧}
\subsubsection{红心实践}
\red{css需要想象力}
\section{HTTP}
\red{Hyper Text Transfer Protocol}
\subsection{URL}
\red{Uniform Resource Locator}
\pl
\red{协议+域名或IP+端口号+路径+查询字符串+锚点}
\subsubsection{IP}
\red{Internet Protocal}
\begin{enumerate}
    \item 约定了两件事
          \begin{itemize}
              \item 如何定位一台设备
              \item 如何封装数据报文,以跟其他设备交流
          \end{itemize}
    \item 外网IP
          \begin{itemize}
              \item 从电信租用带宽,一年一千多。
              \item 买了路由器,然后用电脑和手机分别连接路由器广播出来的无线WIFI。
              \item 路由器连上电信服务器,路由器有一个外围IP,这是你互联网中的地址。
              \item 重启路由器可能会被重新分配外围IP,也就是路由器没有固定的外网IP。
              \item 连接路由器的手机和电脑是内网IP。
          \end{itemize}
    \item 内网IP
          \begin{itemize}
              \item 路由器会在家里创建一个内网,内网设备使用内网IP,一般是192.169.xxx.xxx。
              \item 一般路由器会给自己分配一个好记的内网IP,如192.168.1.1。
              \item 然后路由器会给每一个内网中的设备分配不同的内网IP。
              \item  如电脑是192.168.1.2,手机是192.168.1.3。
          \end{itemize}
    \item 路由器的功能
          \begin{itemize}
              \item 现在路由器会有两个IP,一个外网IP和一个内网IP。
              \item 内网的设备可以互相访问,但是不能直接访问外网。
              \item 内网设备想要访问外围,就必须经过路由器中转。
              \item 外网中的设备可以互相访问,但是无法访问你的内网。
              \item 外网设备想要把内容送到内网,也必须通过路由器。
              \item 也就是说内网和外网就像两个隔绝的空间,无法互通,唯一的联通点就是路由器。
              \item 所以路由器有时候也被叫做网关。
          \end{itemize}
    \item 几个特殊的IP
          \begin{itemize}
              \item 127.0.0.1表示自己。
              \item localhost通过hosts指定为自己。
              \item 0.0.0.0不表示任何设备。
          \end{itemize}
\end{enumerate}
\subsubsection{端口}
\red{一台机器可以提供很多服务,每个服务一个号码,这个号码就叫端口号port}
\begin{enumerate}
    \item 一个比喻
          \begin{itemize}
              \item 麦当劳提供两个窗口,一号快餐,二号咖啡。
              \item 你去快餐窗口点咖啡会被拒绝,让你去两一个窗口。
              \item 你去咖啡窗口点快餐结果一样。
          \end{itemize}
    \item 一台机器可以提供不同服务
          \begin{itemize}
              \item 要提供HTTP服务最好使用80端口。
              \item 要提供HTTPS服务最好使用443端口。
              \item 要提供FTP服务最好使用21端口。
              \item 一共有65535个端口(基本够用)。
          \end{itemize}
    \item 端口使用规则
          \begin{itemize}
              \item 0到1023(2的10次方减1)号端口是留给系统使用的。
              \item 你只有拥有了管理员权限后,才能使用这1024个端口。
              \item 其他端口可以给普通用户使用。
              \item 比如http-server默认使用8080端口。
              \item 一个端口如果被占用,你就只能换一个端口。
          \end{itemize}
\end{enumerate}
\subsubsection{域名}
\begin{enumerate}
    \item 域名就是IP的别称
          \begin{itemize}
              \item baidu.com对应的什么IP --> ping baidu.com
              \item qq.com对应的什么IP --> ping qq.com
              \item 一个域名可以对应不同IP。
              \item 这个叫做均衡负载,防止一台机器扛不住。
              \item 一个IP可以对应不同域名。
              \item 这个叫做共享主机,穷开发者会这么做。
          \end{itemize}
    \item 域名和IP是怎么对应起来的
          \begin{itemize}
              \item 通过DNS
          \end{itemize}
    \item 当你输入qq.com的过程
          \begin{itemize}
              \item 你的Chrome浏览器会向电信提供的DNS服务器询问qq.com对应什么IP。
              \item 电信会回答一个IP(具体过程很发杂,不研究)。
              \item 然后Chrome才会向对应IP的80/443端口发送请求。
              \item 请求内容是查看qq.com的首页。
          \end{itemize}
    \item 为什么是80或443端口
          \begin{itemize}
              \item 服务器默认用80提供http服务。
              \item 服务器默认用443提供https服务。
              \item 可以在开发者工具看到具体的端口。
          \end{itemize}
    \item 题外话
          \begin{itemize}
              \item www.caosiyuan.com和caosiyuan.com不是同一域名。
              \item comn是顶级域名。
              \item caosiyuan.com是二级域名(俗称一级域名)。
              \item www.caosiyuan.com是三级域名(俗称二级)。
              \item 他们是父子关系
              \item 比如github.io把子域名xx.github.io免费给你使用
              \item 但www.caosiyuan.com和caosiyuan.com可以不是同一家公司,也可以是。
              \item www非常多余
          \end{itemize}
    \item 如何请求不同的页面
          \begin{itemize}
              \item 路径可以做到
              \item \url{https://developer.mozilla.org/zh-CN/docs/Web/HTML}
              \item \url{https://developer.mozilla.org/zh-CN/docs/Web/CSS}
              \item 使用chrome开发者工具Network面板看区别。
              \item 有点类似爬虫找规律。
          \end{itemize}
    \item 同一个页面,不同内容
          \begin{itemize}
              \item 查询参数可以做到
              \item \url{http://www.baidu.com/s?wd=hi}
              \item \url{http://www.baidu.com/s?wd=hello}
          \end{itemize}
    \item 同一个页面,不同位置
          \begin{itemize}
              \item 锚点可以做到
              \item \url{https://developer.mozilla.org/zh-CN/docs/Web/CSS#参考书}
              \item \url{https://developer.mozilla.org/zh-CN/docs/Web/CSS#教程}
              \item 注意,锚点看起来有中文,但实际不支持中文。
              \item \# 参考书会变成\#\s{\%E5\%8F\dots}。
              \item 锚点是无法在Network面板看到。
              \item 锚点不会传给服务器。
          \end{itemize}
\end{enumerate}
\subsubsection{HTTP协议}
\red{基于TCP和IP两个协议,规定了请求的格式是什么,响应的格式是什么}
\begin{enumerate}
    \item 用curl可以发HTTP请求
          \begin{itemize}
              \item curl -v http://baidu.com
              \item curl -s -v -- https://www.baidu.com
          \end{itemize}
    \item 理解一下概念
          \begin{itemize}
              \item url会被curl工具重写,先请求DNS获得IP
              \item 先进行TCP连接,TCP连接成功后,开始发送HTTP请求
              \item 请求内容看一眼
              \item 响应内容看一眼
              \item 响应结束后,关闭TCP连接(看不出来)
              \item 真正结束
          \end{itemize}
\end{enumerate}
\subsection{请求响应和NodeJS Sever}



\section{JS}
\subsection{概览}
\red{JS需要一点逻辑能力,数学学的好不用担心,因为比数学简单太多了。}
\subsubsection{硬要求}
\begin{enumerate}
    \item 足够的代码量
          \begin{itemize}
              \item 达到1000行 -->  新手
              \item 达到10000行 --> 熟手
              \item 达到50000行 --> 专业选手
              \item 只能靠时间积累,人生就是奋斗,最快一年就可达到。
          \end{itemize}
    \item 如何统计自己的代码行数
          \begin{itemize}
              \item 安装 yarn global add cloc
              \item 在项目文件下使用 cloc --vcs=git.
              \item 注意把仓库里node\_modules等不想关内容写入.gitignore
          \end{itemize}
    \item 了解最够多的概念,不仅会写,还要会说
          \begin{itemize}
              \item 常用考点:闭包,原型,类,继承,MVC,Flux,高阶函数,前端工程化
              \item 博客总结,代码实践,多多积累。
          \end{itemize}
    \item 有足够的踩坑经验
          \begin{itemize}
              \item 把该领域内所有的错误都犯完的人,就是专家。
              \item 多做个人项目,全方位踩坑。
          \end{itemize}
\end{enumerate}
\subsubsection{JS的历史}
\begin{enumerate}
    \item JavaScript的诞生
          \begin{itemize}
              \item 布兰登生平自行了解,我的总结是成为了领导后千万不能犯错。
              \item 牛逼的程序员不怕辞退,很容易创业,可以干到50岁以上。
              \item 公司要求JS的命名蹭Java的流量,现在各行各业也存在者这种营销。
              \item 由于版权问题,JS又叫ECMAScript。
              \item 布兰登十天设计了JS最初版本(不是实现),所以JS有很多bug。
              \item 网景被微软收购,IE6如日中天。
              \item 2004谷歌雇佣了一些Firefox和IE的开发者。
              \item 2016年Chrome全球份额62\%, 横空出世。
              \item 移动市场智能手机的崛起。
          \end{itemize}
    \item JavaScript的兴起
          \begin{itemize}
              \item 2004年愚人节,谷歌发布Gmail在线网页,当时人们认为网页只能看新闻和图片。
              \item 2005年,Jesse将谷歌用到的技术命名为AJAX,从此,前端技术正式出现。
              \item 用历史唯物主义的观念看,正如现在很多前端概念就是过去技术的打包。
              \item 在此之前的网页开发都是由后端和设计师完成。
              \item 2006年,jQuery发布,是目前最长寿的JS库。
              \item 后来的十年,jQuery大放异彩,直到IE不行了,才稍微没有那么火。
          \end{itemize}
    \item 中国的前端
          \begin{itemize}
              \item 正式出现时间是2010年左右,中国才有专门的前端岗位。
              \item 可以用百度指数关键词搜索趋势。
              \item 早期的前端是一些自学前端的后端程序员,他们把Java思想带入JS
              \item 因此面向对象成了JS的主流思想。
              \item 行业还是很缺前端。
          \end{itemize}
    \item JavaScript的爆发
          \begin{itemize}
              \item Chrome的JS引擎叫做V8, V8原本是跑车引擎的叫法, 快如闪电。
              \item 2009年,Ryon基于V8创建了Node.js。
              \item 2010年,Isaac基于Node.js写出了npm。
              \item 前端工程师可以在浏览器之外执行JS了,Node.js快速风靡。
              \item 同年,TJ受Sinatra启发,发布了Express.js。
              \item 至此,前端工程师可以愉快的写后端应用了。
              \item 至此,爆发了很多技术了,gulp,grunt,yeoman,requireJs,webpack等。
              \item JS是历史的选择,一开始是玩具,但JS走对了风口,所以活到了最后。
              \item 总结:类似考研政治,历史人物可以影响事物的进程,但决定不了历史的发展的方向。
          \end{itemize}
\end{enumerate}
\subsection{内存图与JS世界}
\subsubsection{操作系统常识}
\red{一切都运行在内存里}
\begin{enumerate}
    \item 开机
          \begin{itemize}
              \item 操作系统在C盘里(macOS的在根目录下多个目录里)
              \item 当按下开机键,主板通电,开始读取固件
              \item 固件就是固定在主板上的存储设备,里面有开机程序
              \item 开机程序会将文件里的操作系统加载到内存中运行
          \end{itemize}
    \item 操作系统(以Linux为例)
          \begin{itemize}
              \item 首先加载操作系统内核
              \item 然后启动初始化进程,编号为1,每个进程都有编号
              \item 启动系统服务:文件,安全,联网
              \item 等待用户登录:输入密码登录/ssh登录
              \item 登录后,运行shell,用户就可以和操作系统对话了
              \item bash是一种shell,图形化界面可认为是一种shell
          \end{itemize}
    \item 打开浏览器(chrome.exe)
          \begin{itemize}
              \item 你双击Chrome图标,就会运行chrome.exe文件
              \item 开启Chrome进程,作为主进程
              \item 主进程会开启一些辅助进程,如网络服务,GPU加速
              \item 你每新建一个网页,就有可能会开启一个子进程
          \end{itemize}
    \item 浏览器的功能
          \begin{itemize}
              \item 发起请求,下载HTML,解析HTML,下载CSS,解析CSS
              \item 渲染界面,下载JS,解析JS,执行JS等
              \item 功能模块:用户界面,渲染引擎,JS引擎,存储等
              \item 以上功能模块一般各处于不同的线程(比进程更小)
              \item 如果进程是车间,那么线程就是车间里的流水线
          \end{itemize}
    \item JS引擎
          \begin{itemize}
              \item Chrome用的是由C++编写的V8引擎
              \item 网景用的是SpiderMonkey,后被Firefox使用
              \item Safari用的是JavaScriptCore
              \item IE用的是Chakra(JScript9)
              \item Edge用的是Chakra(JavaScript)
              \item Node.js用的是V8引擎
          \end{itemize}
    \item JS引擎的功能
          \begin{itemize}
              \item 编译:把JS代码翻译为机器能执行的字节码或机器码
              \item 优化:改写代码,使其更高效
              \item 执行:执行上面的字节码或者机器码
              \item 垃圾回收:把JS用完的内存回收,方便之后再次使用
          \end{itemize}
    \item 执行JS代码的准备工作
          \begin{itemize}
              \item 浏览器提供API:window/document/setTimeout
              \item 没错,上面东西都不是JS自身具备的功能
              \item 我们将这些功能称为运行环境runtime env
              \item 一旦把JS放进页面,就开始执行JS
              \item JS代码在内存里运行,看下部分内存图
          \end{itemize}
\end{enumerate}
\subsubsection{内存图}
\red{要求会画内存图}
\begin{figure}[H] %H为当前位置,!htb为忽略美学标准,htbp为浮动图形
    \centering %图片居中
    \includegraphics[width=0.7\textwidth]{image/memory.png} %插入图片,[]中设置图片大小,{}中是图片文件名
    \caption{瓜分内存图} %最终文档中希望显示的图片标题
    \label{瓜分内存图} %用于文内引用的标签
\end{figure}
\begin{enumerate}
    \item 红色区域的作用
          \begin{itemize}
              \item 红色专门用来存放数据,我们目前只研究该区域
              \item 红色区域并不存变量名,变量名在\red{不知什么区}
              \item 每种浏览器的分配规则并不一样
              \item 还有调用栈,任务队列尚未画出
          \end{itemize}
    \item Stack和Heap
          \begin{itemize}
              \item 红色区域分为Stack栈和Heap堆
              \item 栈和堆需要用到数据结构知识
              \item Stack区特点:每个数据顺序存放
              \item Heap区特点:每个数据随机存放
          \end{itemize}
    \item js代码在Heap和Stack区的执行过程
          \begin{lstlisting} 
        var a = 1
        var b = a 
        var person = {name: 'syuancao', hobby: 'coding'}
        var person2 = person
      \end{lstlisting}
          \begin{figure}[H] %H为当前位置,!htb为忽略美学标准,htbp为浮动图形
              \centering %图片居中
              \includegraphics[width=0.7\textwidth]{image/StackAndHeap.png} %插入图片,[]中设置图片大小,{}中是图片文件名
              \caption{js代码在Heap和Stack区执行的内存图} %最终文档中希望显示的图片标题
              \label{js内存图} %用于文内引用的标签
          \end{figure}
    \item 规律
          \begin{itemize}
              \item 数据分两种:非对象和对象
              \item 非对象都存在Stack \red{(数字,字符串,布尔不是对象)}
              \item 对象都存在Heap \red{(数组是对象,函数是对象)}
              \item =号总是会把右边的东西复制到左边 \red{(不存在什么传值和传址,是直接拷贝,图中箭头指向是虚的)}
              \item 很多书上会让你区分值和地址,只有不会画内存图的人才需要做这件事
          \end{itemize}
    \item 对象被篡改,结合内存图很好分析
          \begin{lstlisting} 
        var person = {name:'caosiyuan'}
        var person2 = person
        person2.name = 'syuancao'
        console.log(person.name) // syuancao
      \end{lstlisting}
\end{enumerate}
\subsubsection{JS的世界是怎样的}
\red{神说要有光,就有了光,JS开发者说要有window,就有了window(浏览器提供)}
\begin{enumerate}
    \item JS世界还需要什么
          \begin{itemize}
              \item 要有console,并且挂到window上
              \item 要有document,并且挂到window上
              \item 要有对象,于是就有了Object,并且挂到window上
              \item var person = \{\}等价与 var person = new Object()
              \item 要有数组(一种特殊的对象),于是有了Array,并且挂到window上
              \item var a = [1, 2, 3]等价于 var a = new Array(1, 2, 3)
              \item 要有函数(一种特殊的对象),于是有了Function,并且挂到window上
              \item function f()\{\}等价于 var f = new Function()
          \end{itemize}
    \item 题外话
          \begin{itemize}
              \item 为什么有 var a = [],还要提供var a = new Array()呢
              \item 因为后者是正规写法,但是没人用,前者不正规,但是好用
              \item 为什么有function f()\{\},还要提供var f = new Function写法呢
              \item 原因同上
          \end{itemize}
    \item 把window用内存图画出来
          \begin{figure}[H] %H为当前位置,!htb为忽略美学标准,htbp为浮动图形
              \centering %图片居中
              \includegraphics[width=0.7\textwidth]{image/window.png} %插入图片,[]中设置图片大小,{}中是图片文件名
              \caption{window内存图} %最终文档中希望显示的图片标题
              \label{window内存图} %用于文内引用的标签
          \end{figure}
    \item 更简单的画法
          \begin{figure}[H] %H为当前位置,!htb为忽略美学标准,htbp为浮动图形
              \centering %图片居中
              \includegraphics[width=0.7\textwidth]{image/simplewindow.png} %插入图片,[]中设置图片大小,{}中是图片文件名
              \caption{更简单的window内存图} %最终文档中希望显示的图片标题
              \label{更简单的window内存图} %用于文内引用的标签
          \end{figure}
          \begin{itemize}
              \item 可以用console.dir(window.Array)看属性
              \item 如果第一个字母是大写比如Object,Array,那么会就有prototype属性
          \end{itemize}
    \item 细节
          \begin{itemize}
              \item window变量和window对象是两个东西
              \item window变量是一个容器,存放window对象的地址
              \item window对象是Heap里的一坨数据
              \item 不信的话,可以让var x = window,那么这个x就指向window对象,window变量就可以去死了
              \item 但这样的代码会弄晕新手,所以不要这样写
              \item 但是jQuery就是这样挂到window上的函数,window.jQuery=function()\{\}
              \item 但是jQuery我们平时用\$, 用\$去调用,var \$ = jQuery,\$()
              \item 同理,console(属性)和console对象不是同一个东西
              \item Object和Object函数对象不是同一个东西
              \item 前者是内存地址,后者是内存对应的一坨数据也就是一坨内存
          \end{itemize}
\end{enumerate}
\subsubsection{原型链}
\red{是JS里最重要的,也是新手最难懂的之一(注:JS有三个最难懂的,分别是this,原型,AJAX)}
\begin{enumerate}
    \item 内存图里的prototype是干什么用的
          \begin{itemize}
              \item 可以打印出来看看,console.dir(window.Object.prototype),window可以省略
              \item 只是看起来是一坨无用函数
          \end{itemize}
    \item var obj=\{\} obj.toString()为什么不报错?为什么可以运行?
          \begin{itemize}
              \item obj有一个隐藏属性
              \item 隐藏属性存储了Object.prototype对象的地址
              \item obj.toString()发现obj上没有toString
              \item 就去隐藏属性对应的对象里面找
              \item 于是就找到了Object.prototype.toString里面的toString
              \item 也就是obj.toString === window.prototype.toString
          \end{itemize}
    \item 类似的 var arr=[] arr.join('-')为什么不报错?为什么可以运行?
          \begin{itemize}
              \item arr有一个隐藏属性
              \item 隐藏属性存储了Array.prototype对象的地址
              \item arr.join()发现arr上没有join
              \item 就去隐藏属性对应的对象里面找
              \item 于是就找到了Array.prototype.join里面的join
              \item 也就是Array.prototype.join === window.prototype.join
          \end{itemize}
    \item JS的光
          \begin{itemize}
              \item 下面一张图可以解释上面的问题
                    \begin{figure}[H] %H为当前位置,!htb为忽略美学标准,htbp为浮动图形
                        \centering %图片居中
                        \includegraphics[width=0.7\textwidth]{image/thelightOfJs.png} %插入图片,[]中设置图片大小,{}中是图片文件名
                        \caption{JS的光} %最终文档中希望显示的图片标题
                        \label{js的光} %用于文内引用的标签JS
                    \end{figure}
          \end{itemize}
    \item var obj2=\{\} obj2.toString() obj和obj2有什么联系
          \begin{itemize}
              \item 相同点:都调用.toString
              \item 不同点:地址不同obj!==obj2, 可以拥有不同的属性
              \item xxx.prototype存储了xxx对象的共同属性,这就是原型
          \end{itemize}
    \item 原型的好处
          \begin{itemize}
              \item 如果没有原型,声明一个对象
                    \begin{lstlisting} 
                    var obj = {
                        toString: window.Object.prototype.toString,
                        hasOwnPropertyof: window.Object......
                    }
                    obj.toString() 
                    var obj2 = {
                        toString: window.Object.prototype.toString,
                        hasOwnPropertyof: window.Object......
                    }
                    obj2.toString() 
      \end{lstlisting}
              \item 你是不是想累死自己
              \item 原型让你无需重复声明共有属性,省代码,省内存
          \end{itemize}
    \item 关于隐藏属性\_proto\_
          \begin{itemize}
              \item 每个对象都有一个隐藏属性,用来保存其原型的地址,这个隐藏属性的名字叫做\_proto\_
              \item 大写的不要关心隐藏属性,这涉及js哲学问题不用关心,关心小写的隐藏属性
              \item 如果没有隐藏属性,obj就不知道共有属性在哪,就没把法调用toString等
          \end{itemize}
    \item prototype和\_proto\_的区别是什么
          \begin{itemize}
              \item 都存着原型的地址,即相同的地址
              \item 只不过prototype挂在函数上,通常是大写的(Array, Object, Function)上面
              \item \_proto\_挂在每个新生成的对象上,也就是小写的(var a = \{\}, var b = [])上面
          \end{itemize}
    \item 犀利的提问
          \begin{itemize}
              \item 类似之前提过篡改对象的例子
                    \begin{lstlisting} 
                        var obj = {}
                        var obj2 = {}
                        obj.toString === obj2.toString //输出为true
                        obj.toString = 'fuck' // 输出为"fuck"
                        obj.toString //也等于 'fuck'吗?
              \end{lstlisting}
              \item 不废话一图解决问题
                    \begin{figure}[H] %H为当前位置,!htb为忽略美学标准,htbp为浮动图形
                        \centering %图片居中
                        \includegraphics[width=0.7\textwidth]{image/objtoString.png} %插入图片,[]中设置图片大小,{}中是图片文件名
                        \caption{toString篡改} %最终文档中希望显示的图片标题
                        \label{toString篡改} %用于文内引用的标签
                    \end{figure}
              \item 原因解释
                    \begin{itemize}
                        \item 这个和之前的不一样,这里toString是隐藏属性
                        \item 结合上图相当于写了两层
                        \item 所以一层是可以篡改的,两层就不可以
                    \end{itemize}
          \end{itemize}
\end{enumerate}
\subsection{Canvas实践--画图板}
\red{\href{https://github.com/syuancao/FrontNotePractice/tree/master/JS/0-Canvas}{项目地址} }
\pl
\red{\href{https://syuancao.github.io/FrontNotePractice/JS/0-Canvas/index.html}{预览效果} }
\subsection{JS语法}
\red{es6是最低要求}
\subsubsection{JS版本}
\begin{enumerate}
    \item 历史版本
          \begin{itemize}
              \item ES3,IE6支持,总体评价:垃圾
              \item ES5,总体评价:还是垃圾
              \item ES6,大部分浏览器支持,总体评级:一半垃圾一半好
              \item ES2019与ES6差别不大
          \end{itemize}
    \item 为什么说ES6一半垃圾
          \begin{itemize}
              \item 因为ES不能删除以前的特性,需要兼容旧网站
              \item 也就是说以前能运行的网站,以后都要能运行
              \item 对比Python3你就能知道兼容的好处:稳定
          \end{itemize}
    \item 一门语言的价值
          \begin{itemize}
              \item 是由其产生的价值决定
              \item JS是世界上使用最广的语言
              \item JS是门槛极低的语言(只要你不学糟粕)
              \item JS是一门能产生价值的语言(虽然不美)
              \item 它的优秀之处并非原创,它的原创之处并不优秀
          \end{itemize}
\end{enumerate}
\subsubsection{语法}
\begin{enumerate}
    \item 表达式
          \begin{itemize}
              \item 1+2表达式的值为3
              \item add(1,2)表达式的值为函数的返回值
              \item console.log表达式的值为函数本身
              \item console.log(3)表达式的值为多少?(undefined)
          \end{itemize}
    \item 语句
          \begin{itemize}
              \item var a = 1是一个和语句
          \end{itemize}
    \item 二者的区别
          \begin{itemize}
              \item 表达式一般都有值,语句可能有也可能没有
              \item 语句一般会改变环境(声明,赋值)
              \item 上面两句话并不是绝对的
          \end{itemize}
    \item 大小写敏感
          \begin{itemize}
              \item var a和var A是不同的
              \item object和Object是不同的
              \item function和Function是不同的
              \item 具体含义后面说
          \end{itemize}
    \item 空格
          \begin{itemize}
              \item 大部分空格没有实际意义
              \item var a = 1和var a=1没有区别
              \item 加回车大部分时候也不影响
              \item 只有一个地方不能加回车,那就是return后面
          \end{itemize}
    \item 标识符的规则
          \begin{itemize}
              \item 第一个字符,可以是Unicode字母或\$或\-或中字
              \item 后面的字符,除了上所说,还可以有数字
              \item 变量名是标识符
          \end{itemize}
    \item 区块block
          \begin{itemize}
              \item 把代码包在一起
              \item 常常与if/for/while合用
          \end{itemize}
    \item if语句
          \begin{itemize}
              \item if(表达式){语句1}else{语句2}
              \item \{\}在语句只有一句的时候可以省略,但不建议这么做
          \end{itemize}
    \item if语句变态情况
          \begin{itemize}
              \item 表达式里可以非常变态,如 a = 1
              \item 语句1里可以非常变态,如嵌套if else
              \item 语句2里可以非常变态,如嵌套if else
              \item 缩进也可以很变态,如面试常常下套
                    \begin{lstlisting} 
                        a = 1
                        if (a === 2) 
                            console.log('a')
                            console.log('a等于2')
                    \end{lstlisting}
          \end{itemize}
    \item while循环
          \begin{itemize}
              \item while(表达式){语句}
              \item 当表达式为真,执行语句,执行完再判断表达式真假
              \item 当表达式为假,执行后面的语句
              \item do ... while可以做用户输入判定
          \end{itemize}
    \item for循环
          \begin{itemize}
              \item for循环是while的方便写法
              \item for(语句1;表达式2;语句3) {循环体}
              \item 不多说了,和C语言一样
          \end{itemize}
    \item label语句
          \begin{itemize}
              \item 语法
                    \begin{lstlisting} 
                foo: {
                    console.log(1);
                    break foo;
                    console.log('本行不会输出);
                }
                console.log(2);

                foo: 1; 

                {
                    foo: 1;
                }
            \end{lstlisting}
              \item 面试:下面的东西是什么?下面是代码块,是label,不是对象
                    \begin{lstlisting} 
                        {
                            foo: 1
                        }
            \end{lstlisting}
          \end{itemize}
\end{enumerate}
\subsubsection{JS数据}
\begin{enumerate}
    \item 7种数据类型(大小写无所谓)
          \begin{itemize}
              \item 数字number
              \item 字符串string
              \item 布尔bool
              \item 符号symbol
              \item 空undefined
              \item 空null
              \item 对象object
              \item 总结:四基两空一对象
          \end{itemize}
    \item 以下不是数据类型
          \begin{itemize}
              \item 数组,函数,日期
              \item 它们都属于object
          \end{itemize}
    \item 5个falsy值
          \begin{itemize}
              \item falsy就是相当于false但又不是false值
              \item 分别是undefined null 0 NaN ''
              \item '' '  ' 即空字符串和空格字符串不是一个玩意
          \end{itemize}
    \item undefined和null的区别
          \begin{itemize}
              \item 这是js的垃圾之处,没有本质区别
              \item 如果一个变量声明了,但没有赋值,那么默认值就是undefined而不是null
              \item 如果一个函数,没有写return,那么默认return undefined, 而不是null
              \item 前端程序员习惯上,把非对象的空值写成undefined,把对象空值写为null
              \item 仅仅是习惯而已
          \end{itemize}
    \item let声明
          \begin{itemize}
              \item 遵循块作用域,即使用范围不能超出{}
              \item 不能重复申明
              \item 可以赋值,也可以不赋值
              \item 必须先声明再使用,否则报错
              \item 全局声明的let变量,不会变成window的属性
              \item for循环配合let有奇效
          \end{itemize}
    \item 类型转换
          \begin{itemize}
              \item number => string
                    \begin{itemize}
                        \item String(n)
                        \item n + ''
                    \end{itemize}
              \item string => number
                    \begin{itemize}
                        \item Number(s)
                        \item parseInt(s)/parseFloat(s)
                        \item s - 0
                    \end{itemize}
              \item x => bool
                    \begin{itemize}
                        \item Boolean(x)
                        \item x.toString()
                    \end{itemize}
          \end{itemize}
\end{enumerate}
\subsubsection{JS对象}
\red{第七种数据类型,唯一一种复杂类型}
\begin{enumerate}
    \item 定义
          \begin{itemize}
              \item 无序的数据集合
              \item 键值对的集合
          \end{itemize}
    \item 写法

          \begin{lstlisting} 
                let obj = {'name': 'caosiyuan', 'aga': '27'}
                let obj = new Object({'name': 'caosiyuan'})
                console.log({'name': 'caosiyuan', 'age': 18})
            \end{lstlisting}

    \item 细节
          \begin{itemize}
              \item 键名是字符串,不是标识符,可以包含任意字符
              \item 引号可以省略,省略之后就只能写标识符
              \item \red{就算引号省略了,键名也还是字符串}
          \end{itemize}
    \item 变量作属性名
          \begin{itemize}
              \item 不加[]的属性名会自动变成字符串
              \item 加了[]则会当作变量求值
              \item 值如果不是字符串,则会自动变成字符串
              \item 除了字符串,symbol也能做属性名
          \end{itemize}
    \item 对象的隐藏属性
          \begin{itemize}
              \item JS中每一个对象都有一个隐藏属性
              \item 这个隐藏属性储存着其\red{共有属性组成的对象}的地址
              \item 这个\red{共有属性组成的对象}叫做原型
              \item 也就是说,隐藏属性储存着原型的地址
          \end{itemize}
    \item 删除属性
          \begin{itemize}
              \item delete obj.xxx或obj['xxx']
              \item 即可删除obj的xxx属性,请区分属性值为undefined和不含属性名
              \item 不含属性名 'xxx' in obj === false
              \item 含有属性名,但是值为undefined,'xxx' in obj \&\& obj.xxx === undefined
              \item 注意obj.xxx === undefined不能断定'xxx'是否为obj的属性
          \end{itemize}
    \item 查看所有属性(读属性)
          \begin{itemize}
              \item 查看自身所有属性Object.keys(obj)
              \item 查看自身+共有属性console.dir(obj)
              \item 或者自己依次用Object.keys打印出obj.\_proto\_
              \item obj.hasOwnPropertyof('toString')判断一个属性是自身的还是共有的
          \end{itemize}
    \item 原型
          \begin{itemize}
              \item 每个对象都有原型,原型里存着对象的共有属性
              \item 比如obj的原型就是一个对象
              \item obj.\_proto\_存着这个对象的地址
              \item 这个对象里有toString/constructor/valueOf等属性
              \item 对象的原型也是对象,所以对象的原型也有对象
              \item obj={}的原型即为所有对象的原型
              \item 这个原型包含所有对象的共有属性,是对象的根
              \item 这个原型也有原型,是null
          \end{itemize}
    \item 查看属性
          \begin{itemize}
              \item 中括号语法:obj['key']
              \item 点语法:obj.key
              \item 坑新人语法:obj[key] // 变量key值一般不为'key'
              \item 优先使用中括号语法,点语法会误导你,让你以为key不是字符串
          \end{itemize}
    \item 修改或增加属性(写属性)
          \begin{itemize}
              \item 直接赋值 obj.name = 'caosiyuan'
              \item 批量赋值 Object.assign(obj, {age: 27, gender: 'man'})
          \end{itemize}
    \item 修改或增加共有属性
          \begin{itemize}
              \item 无法通过自身修改或增加共有属性
              \item let obj = {}, obj2 = {}//共有toString
              \item obj.toString = 'xxx'只会再改obj自身属性
              \item obj.toString还是在原型上
          \end{itemize}
    \item 我偏要修改或增加原型上的属性
          \begin{itemize}
              \item obj.\_proto\_toString = 'xxx'//不推荐用\_proto\_
              \item Object.prototype.toString='xxx'
              \item 一般来说,不要修改原型,这会引起很多问题
          \end{itemize}
    \item  修改隐藏属性
          \begin{itemize}
              \item  不推荐用\_proto\_
                    \begin{lstlisting} 
                let obj = {'name': 'caosiyuan'}
                let obj2 = {'name': 'syuancao'}
                let common = {kind: 'human'}
                obj.__proto__ = commmon 
                obj2.__proto__ = common
            \end{lstlisting}

              \item 推荐使用Object.create
                    \begin{lstlisting} 
                        let obj = Object.create(common)
                        obj.name = 'caosiyuan'
                        let obj2 = Object.create(common)
                        obj2.name = 'syuancao'
                        
            \end{lstlisting}

          \end{itemize}
    \item 'name' in obj和obj.hasOwnProperty('name') 的区别
          \begin{itemize}
              \item  'name' in obj会检查对象隐藏属性即原型链
              \item obj.hasOwnProperty('name')只会检查自身,即该属必须是对象本身的成员
          \end{itemize}
\end{enumerate}
\section{算法与数据结构}
\section{JS编程接口}
\section{项目}
\section{MVC}
\section{Webpack}
\section{虚拟DOM与DOM diff}
\section{Vue}
\section{React}
\section{NodeJS}
\section{Vue3造轮子}

\end{document}